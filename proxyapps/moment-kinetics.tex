\newsection{Moment\_Kinetics}{sec:MomentKinetics}

The moment kinetics proxyapp solves the kinetic-type equations derived in the 
Oxford project (Parra et al.). The proxyapp prototypes three features:

The code implements a ``moment kinetics" approach to kinetic equations, where 
the particle distribution function is modified to remove the density, parallel 
bulk velocity and/or temperature. These fluid moments are then evolved 
separately from the distribution function. This approach allows a simulation to 
dynamically switch between a kinetic and a fluid model.

The code is implemented using a Chebychev spectral element approach in both 
physical space and velocity space. This gives spectral convergence with grid 
resolution while still allowing the flexibility to model complicated physical 
domains. For comparison, the code is also implemented using a standard finite 
difference approach.

The code is implemented in the Julia language. Thus by leveraging Julia 
community packages, the code can execute on CPUs and GPUs with minimal input 
from the developer (e.g. no explicit domain decomposition between MPI ranks). 
This may provide an alternative approach to performance portability, as opposed 
to the current expected approach of using C++ with code generation and 
domain-specific languages (DSLs).

%Julia implementation of moment-kinetic approach to non-equilbrium plasma 
dynamics.

Moment\_Kinetics is publicly available at 
\url{https://github.com/mabarnes/moment_kinetics}.
