\subsection{Geometry}\label{sec:geometry}

\begin{quote}
``Ability to handle complex topologies and novel geometries:
\begin{itemize}
\item Capacity to treat different topologies with multiple X-points ($>$2),
	possibly in a dynamic way, or at least implemented in a way that does
	not preclude time-varying equilibria being modelled in future; 
\item Ability to handle singularities in the grid (null points) or to develop
	accurate scheme(s) that do not have singularities;
\item Ability to interface with 3D CAD designs of the machine; 
\item Capability to handle conformal grids that go all the way to the wall for
	both neutrals and plasma; 
\item Capability to treat subdivertor structures (only neutrals).''
\end{itemize}
\end{quote}

\nep\ will use a spectral/hp finite element approach.
This allows considerable flexibility in meshing, with elemental tetrahedra,
hexahedra or prisms being used to describe complex geometries.
In addition to refinement of the elemental grid sizes ($h$-adaptivity),
the order of the piecewise polynomial basis functions used on each element may
also be changed ($p$-adaptivity).
The production of finite element griddings conforming to CAD descriptions is standard
for virtually all meshing packages. Project \nep \ will specially aim to use
the small number of packages that may produce finite elements with curved edges and faces 
to exploit fully the higher order basis.
This gives a framework which can achieve spectral convergence on complicated, conforming grids,
while still being able to handle discontinuities in the solution.

%Unlike approaches based on deformations of Cartesian grids
%{\blue (JTP: is this what is meant by conformal mappings here?)},
Should it prove necessary to produce a meshing conforming to
surfaces of an equilibrium magnetic field, then because
the finite element approach does not require a structured grid, it should
be able to treat arbitrary geometries and topologies, 
in particular including multiple X-points.
The approach could be extended by further use of adaptive meshing to allow for time-varying equilibria.

%From the science plan
%\begin{quote}
%	``[NEPTUNE] work will initially focus upon coupling the turbulent plasma periphery
%	to the surrounding neutral gas and partially ionized impurities that
%	exist between the plasma and plasma facing components, in the presence
%	of an arbitrary tokamak magnetic field and full 3-D first wall
%	geometry.''
%
%	``Infrastructure and workflows will be developed so that the resulting
%	close coupled models can be constructed routinely based around a
%	high-fidelity representation of the geometry described by Computer
%	Aided Design (CAD) systems (this likely requiring the development of
%	high order meshing technology, e.g.\ using Nektar++).''
%
%	``P4. Accurate representation of first wall geometry (face normals to
%	within 0.1$^\circ$), and correspondingly of complex magnetic field
%	geometries.''
%
%	Solvers in complex geometries part of ``Plasma multiphysics model
%	(Fusion modelling, work package FM-WP2)''
%
%\end{quote}
