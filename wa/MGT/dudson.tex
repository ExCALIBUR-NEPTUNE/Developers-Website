\newsection{Management}{sec:MGT_MGT}

Meetings, whether on-line or in-person are regarded as critical for good collaboration,
and are discussed in \Sec{meet}. The other key collaborative element centres
naturally on the software, where use of the \T{git} control system, see \Sec{version}
and consequent use of repositories, see \Sec{repo}, is becoming universal.

\subsection{Meetings and Workshops} \label{sec:meet}

To start the project and any notable identifiable piece of work
within it, a kick-off meeting should bring
together all partners who will contribute significant code to the
project.  The aim of this meeting will be to build personal links
among the team, and to establish community practices consistent with the charter.
Efforts should be made to build consensus and
a community spirit within the project team.

A regular project planning and monitoring meeting should be set up,
at least monthly. Its agenda should include short updates on progress of
each project component, and focus on the project planning and
coordination. In addition, a separate series of seminars and training
should be organised, where each partner might give a longer talk on an
aspect of their work, for example showing other partners how to use
recently developed capabilities.


Development and collaboration mechanisms should include:
\begin{enumerate}
\item A system of code repositories for version control (eg.\ \T{github})
\item Automated testing infrastructure (eg.\ \T{github} actions)
\item Documentation infrastructure, ie.\ as a website
\item A repository for long-term storage of large files, records of
meetings, presentations etc. (eg.\ Google shared drive)
\item A chat/messaging service such as Slack or Zulip, to facilitate interactions
between developers
\end{enumerate}
As these are established, a series of training workshops should be
arranged. These should include talks on the ``high level" objectives, 
on the near-term plans of each partner, and also hands-on training in
the tools being used. 


\subsection{Version control} \label{sec:version}

The standard \T{git} version control system should be used; there is no
viable competitor to this in terms of capabilities, widespread adoption,
or integration into other tools and services (eg.\ \T{github}).

A common complaint against \T{git} is the user interface, which can be intimidating 
to new users. There are very strong reasons why even programmers with
plenty of other experience, should seek guidance and preferably training
in use of the command line interface~(CLI). For those who have time enough to attempt to do so without, 
a few hints are provided: 
\begin{enumerate}
\item The complexity of the interface can be mitigated by restricting usage
to a few well-chosen subcommands such as \T{clone}, \T{add}, \T{commit}, \T{push}, \T{pull}, \T{diff}, \T{log} and \T{status}.
\item  Exercise caution before using other
subcommands or new options to the core subcommands, eg.\  by first committing all files, adding a suboption
which indicates what will be done without actually modifying any files, and avoiding 
forcing options.
\item For the purpose of the key subcommands such as `pull' and `push', it is important
to remember than these are are defined from the user's point-of-view,
so that `pull' brings source from the repo towards the user, and `push' sends it away.
There are other non-intuitive aspects so that it is important to study very carefully the
description of any new sub-command/option and particularly its ordering of options.
\item Since the software is widely used, error messages can invariably be `googled'
for further explanation.
\item Should conflicts occur, these are recorded by the insertion of strings `+++\ldots',
`\verb->>>>...-' and `\verb-<<<<...-' in disc files to indicate lines where the clashes lie. Many users
find resolving conflicts very difficult on the basis of such information,
however making up for  the absence of a GUI mechanism within \T{git} to do this,
it is possible to integrate GUIs such as \B{meld}, being aware of possible system dependences.
\end{enumerate}

Otherwise, the experience of \T{git} can be mitigated through:
\begin{itemize}
\item Training: Links to training material for adopted tools should be made available
as part of the project documentation. This should be supplemented by training,
both one-to-one and as part of a programme of talks and training.

\item Adoption of, and training in, tools to provide easier interfaces. \T{github} itself
allows browsing of history; \T{Magit} is an excellent interface integrated into Emacs;
and similar tools exist for eg.\ Visual Studio Code.
The ITER organisation uses \T{bitbucket} and UKAEA uses \T{gitlab}.
\end{itemize}
%Training in use of \T{git} should make clear that pretty every much aspect of the
%CLI is unsatisfactory and similar features should be deprecated in the new development. 


\subsection{Code repositories} \label{sec:repo}

The structure of \exc \  will result in a number of different
components, experimental \papp s, and increasingly complex
applications. There are two main different approaches as to how these different
components could be split between git repositories, namely (1) Several large code
bases are kept in a single repository (a `monorepo') and (2) 
projects are kept in  separate repositories, with
dependencies being included as \T{git} submodules. 

%Advantages of `monorepos' include simplified dependencies, synchronisation of
%changes to different components, and a central location for documentation.
%The main argument for having separate repositories is for the occasion when several components
%already pre-exist as separate repositories. For new modules, strong
%coupling between components should be discouraged, so that components
%can be reused in a range of applications. %The separate repositories
%enables finer control over access rights, though this can imply
%additional barriers to shared code ownership.

Adopted is a compromise approach whereby:
\begin{itemize}
\item A \T{github} `organisation' \url{https://github.com/ExCALIBUR-NEPTUNE}
was created to host new repositories. 
%(The names ``Neptune" and ``Excalibur" were taken, but
%``excalibur-neptune" was available, for example.)
Organisations allow permissions for groups of administrators and
developers to be managed, and this is currently restricted to community members, so
that it is important to be `logged in' to access their components.
\item Individual components and \papp s are hosted in separate repositories under this
organisation. These contain the code, unit tests, documentation etc.
specific to these components.
\item Reports produced as part of the \nep \ project, unless they
contain commercially sensitive information, are to be found in the
repository~\cite{xpndocswebsite}.
\item A central repository under this 
organisation includes components as sub-modules. These could be
organised into a directory structure, with documentation explaining the relations
or coupling between components. In this repository should go:
\begin{itemize}
\item Integration tests which couple components and ensure that they work together
\item Documentation of the interfaces between components, project conventions
(eg.\ style guides), and overall project aims.
\end{itemize}
Sub-modules are pinned to a particular \T{git} commit, so that at any
point the versions included are those which are known to work with
each other. A developer who wants the latest version of a component
should clone the individual repository, while a user who wants
something that ``just works" should clone the central repository.
\end{itemize}
(There is the disadvantage of a tie specifically to \T{github}, but loss
of the `organisation' capability would be expected to be an inconvenience rather
than a disaster for a project.)

\subsection{Development workflow} \label{sec:develop}

The standard \T{git} work flow has been adopted, since this is widely
familiar and has been developed as best practice based on industrial
experience. Exceptions are allowed for minor issues, such as typographical
errors and broken links in documentation.

Each code component maintains a \T{main} branch (often referred to as the `master'
as in `master copy'), which can only be modified
through a pull request mechanism which ensures peer review and testing. Bug fixes
and feature development are done in separate branches, either in the same
repository, or in forked repositories. When someone encounters a bug, or wishes to
develop a new feature, the recommended approach is:

\begin{enumerate}
\item An issue is opened, describing the bug or feature request or proposal. This
allows discussion of the issue, and possible approaches to addressing it.
\item A pull request is opened as early as possible, marked ``Work in
progress" or similar. This can contain only minimal code or outline
of the code structure. This links to the issue, lets other people
know that it is being worked on, and enables peer review and input
into the development direction.
\item Once ready for merging, and consensus has been reached that the proposed change
should be made, then it is merged.
\end{enumerate}

If a code is sufficiently large, then a further degree of separation between
the stable \T{main} branch and active development is needed. A common pattern
is to only branch off and merge features into a  \T{next} branch. Periodically
this branch is merged into \T{main} as a new release, once the new features 
are judged to be sufficiently mature and tested.

Whether into \T{main} or \T{next}, pull requests should be reviewed
using a checklist the remind the reviewers. 
Review involves testing, aspects of which are addressed in \Sec{test}.

It must be stressed that code review is not a job separate from
code development: {\bf All developers should be expected to participate in
and carry out code reviews}. Reviewing code benefits not only the
original author, but also the reviewer. Through the discussion, it
contributes to a sense of shared ownership of the code base, and
spreads good practices. There is the implication that code should be
written `for the other guy', ie.\ so that the other guy can understand it
without much difficulty.  It also ensures that at least two developers
know how each part of the code works.

\subsection{Code release}

Code releases should be a regular occurrence.
Code release helps with project branding and user engagement,
and ensures that the project is seen as active.
It also helps project administration by ensuring new features are shared in a timely fashion,
and by reducing the number of long-lived divergent branches. 

The project \nep\ codebase contains \papp s, and infrastructure code that interfaces them.
A code release will consist of a version of this infrastructure code, 
plus commit hashes that fix the versions of the \papp s.
As \papp s might be independent projects with their own established release cycle,
the following release policy applies only to the infrastructure code.
It is the recommended policy for new \papp s written under Project \nep.

Release numbering should follow (a modified) Semantic Versioning approach~\cite{semverwebsite}, summarized as
\begin{quote}
``Given a version number MAJOR.MINOR.PATCH, increment the:
\begin{enumerate}
  \item MAJOR version when you make incompatible API changes,
  \item MINOR version when you add functionality in a backwards compatible manner, and
  \item PATCH version when you make backwards compatible bug fixes.''
\end{enumerate}
\end{quote}
Here it is understood that ``API'' refers to user-facing interfaces;
APIs to functions internal to \papp s may break backwards compatibility in
MINOR releases.
There is however an absolute guarantee that no backwards incompatible changes
are made for end users in MINOR and PATCH releases, except those that arise
from fixing a bug.
That is, physics results are permitted to change in such releases if the new
release's results are ``correct'' and the previous release's results were
``wrong''.

It is also understood that releases with MAJOR number 0 are considered beta releases,
for which there are no guarantees of backwards compatibility.
 
Each release will be uploaded as a Zenodo\cite{zenodowebsite}
record with its own DOI.
This gives a clear citation for the project (to be included in the project's
README or CITATION.cff file),
while ensuring that developers receive credit for their work, without the need
for associating each release with a publication.
Encouraging researchers to use release versions and to cite by version number
also aids scientific reproducibility.

While some technical aspects of the release process can be automated,
many of the tasks, such as curating issues and writing release notes, are
inherently manual.
To prevent \nep\ relying on a single person to make releases,
the exact workflow will be codified and included in the project documentation.
An example of such a workflow for the GS2 project may be found
online~\cite{gs2website}.


