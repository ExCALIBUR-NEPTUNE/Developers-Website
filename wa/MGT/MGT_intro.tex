\newsection{Introduction}{sec:MGT_intro}
Webpages on this site concerning management details are taken from the
report~\cite{y3re314}, based largely on the work of Ben Dudson, which
presents guidance concerning the mechanics of an opensource development
by a community distributed  across different sites and organisations,
intended to produce software intended for widespread long-term usage.
Parts of~\cite{y3re314} relate more to 
the technical specification~(TS) and are reproduced in \Sec{TS_sw_response},
another part concerns operational aspects~(OP), see \Sec{OP_MGT}.
Its recommendations are broadly consistent with those laid out by Bungarth \&
Heister~\cite{Ba13What}, and in particular those for the usage of \T{git}
conform to practice recommended by the ITER organisation.
This document is not the place for a general discussion of software engineering practices,
and does not cover code coupling, both of which topics are discussed in the open literature,
see in particular Lawrence~et~al~\cite{La18Cros} for HPC software engineering and
Belete~et~al~\cite{Be17over} for code coupling,
also see other \nep \ reports, particularly~\cite{y2re312,y2re333,y3re72}.

\cite{y3re314} assumes that all the community has signed up to a ``Charter"
which for \nep \ appeared as report~\cite{charter}, reproduced in \Sec{charter}.
The guidance\cite{y3re314}  includes  important issues that need to be agreed as early as possible.
%drawing on the ``Development Plan" document~\cite{y2d34} and on
These include practical points concerning frequency of meetings, code review etc.,
designed to ensure efficient collaboration between a wide group of project partners. 
%It is important to distinguish use-cases. There appear to be three
%use-cases worth treating separately:
%\begin{enumerate}
%\item code produced for immediate and local use only, eg\. to test out an idea or to illustrate a
%scientific paper.
%\item software for long-term use, where execution speed is time-critical, eg\. for real-time
%control or inter-shot discharge analysis.
%\end{enumerate}
%This note relates most closely to the third case.
The guidance document\cite{y3re314} also seeks not merely to prescribe, but to
give compelling arguments for the choices made in respect of guidelines.
Acknowledging the possibility of disagreements, it states that efforts will be made to ensure
consensus or at least agreement between the two most affected project partners
on any decisions taken. However, in the event of continuing disagreement, the technical
leader or `Lead' for the project will ultimately decide on the basis of technical
evidence presented, subject to ratification by higher management.

One general rule is always to allow two options (`rule of two'), intended to enable exploitation 
and possible incorporation of any promising new software (eg.\ package, library or language)
or relevant algorithm which emerges during the course of the project.
Since however, each option doubles the potential cost of developing and maintaining software,
a good case must be made to the Lead for a new option, and the innovator include provision
for retiring one of the existing options should there already be two. Implicitly
thereby, as discussed at the end of \Sec{lang}, a third exploratory option is also allowed.

A similar recommendation (rather than rule) regarding both code and documentation is to
`write once, re-use many times'.
This to a large extent explains a preference for the \LaTeX \ {\tt lwarp} package as enabling
multiple reuse of the same text and mathematical expressions in different
documents and on different webpages.


%The guidelines for a development are set out in the body of this document in an
%arrangement consistent with the concordance, such that separate sections
%correspond to the different documents/web-pages required.

