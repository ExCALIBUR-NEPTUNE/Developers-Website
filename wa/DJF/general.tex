A prime consideration for edge modelling is accurately to represent the surface
normal, not just the surface itself, since good power-handling demands that
plasma flows in close to tangency, typically within $2^{o}$. Unless there is to be
special coding at boundaries, the educational material for finite difference
modellers explains why this approach is unsatisfactory, and the same objection
also applies to AMR type meshes, although in view of the unexpected success of AMR
in fluid and PIC codes, it was carefully discussed as detailed in ref~\cite{y1re211}. The
most serious objection appeared to be the need to represent plasma diffusion
which is strongly anisotropic due to an applied magnetic field. (Moreover,
if hanging nodes are allowed in the FEM, then AMR behaviour may be reproduced by FEM codes,
whereas the converse is not true.)

Similar objections also apply to Lattice Boltzmann methods~(LBM), where there is the simple
objection that since their complexity arises from treating diffusive transport, why not code
the diffusive operators directly? Equally when it is significant that
the dissipation operator is mathematically extremely complex as
in many plasma transport problems, why not use a more accurate representation of
velocity-phase space effects than is provided by a small number of LBM constants?
Although proponents claim that apparently fundamental inefficiencies in the approach,
such as the need for explicit time advance, may be overcome, the ideas needed have
invariably been already explored in the finite volume or finite element context,
so as LBM approaches the efficiency of say FEM, it increasingly resembles it algorithmically.

Smoothed particle hydrodynamics~(SPH) may be eliminated as a general approach to
plasma modelling on purely mathematical grounds, namely that for small dimensionality
problems it is provably more accurate to use a mesh-based approximation to reconstruct a function
than a set of point samples, see for example Niederreiter's textbook~\cite{niederreiter}.
Exactly where the bound lies depends on details of the problem, but it seems in
practice to be around $D=4$, so that particle methods may become attractive for
treating kinetic effects in $5-D$~or~$6-D$ position-plus-velocity phase spaces.
Notice that SPH `particles' are really just markers for mass, momentum and energy,
which furnishes a physics-based argument against their use, since 
if additional (kinetic) effects are to be introduced into a problem such
markers are not the best starting-point for inclusion.

Both SPH and LBM may have niches when mesh production is difficult, such as strong
surface deformation cracks, however these niches do not seem presently to need
occupation in order to design a fusion reactor. Gentler deformations due to thermal
expansion and melt may be handled by moving finite element methods which have been
introduced into existing FEM packages as required. Moreover, if indeed rapid
boundary changes are concerned, the envisaged combined particle and mesh \nep\ 
software would anyway seem to offer an excellent foundation for handling the whole
range of shape variation.

Lastly, `particle shaping' has been mentioned in the context of particle methods
(regarded as methods directly capturing some aspect of velocity space, ie.\ not SPH).
To most mathematicians and physicists, giving a particle a `shape' would seem unnatural,
and the only benefit that it seems to offer, namely a reduced level of noise,
is more efficiently handled by filtering in physical space alone,
in the \nep\ context by smoothing any related mesh-based quantities.



