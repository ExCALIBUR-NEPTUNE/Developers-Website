For Activity 2, UKAEA produced three milestone reports, namely
\begin{itemize}
\item CD/EXCALIBUR-FMS/0012-M2.1.1  - Options for Geometry Representation 
\item CD/EXCALIBUR-FMS/0013-M2.3.1  - Options for Particle Algorithms
\item CD/EXCALIBUR-FMS/0031-M2.5.1  - Select techniques for MOR (Model Order Reduction)
\end{itemize}
together with the Activity~1 Report ('Equations document') \\
CD/EXCALIBUR-FMS/0021-1.00-M1.2.1 - Equations for \nep/\exc \  \papp s

The first two (Reports 12 and 13) describe the problems presented by the fusion use case
in respect of geometry and strong magnetic field in the first, and of generally but
not invariably low collisionality of the edge plasma in the second.  They set out issues
that needed to be urgently addressed and possible lines of research.
Report~31 discusses in depth a wide range of options for research into
Model Order Reduction, drawing attention to the possibility of producing scalable
algorithms by borrowing ideas from the field of Data Assimilation.
Report~21 sets out equations to be studied using the first six \papp s,
beginning with relatively simple models  for anisotropic transport and advancing to
complex models of plasma-neutral interaction. 

%Report~21 was distributed with all the calls issued, and Report~13 was distributed with the call T/NA079-20.
%Report~12 has been used as the basis for extensive discussions with the winner of bid T/NA078-20.
%It is expected that all the reports, including Report~26 will form a useful background to
%interactions with grantees throughout 2021. The first three months of these  interactions have already
%shaped the Y3~plan causing it to have a significant particles element as Activity~4 (Report~13)
%a significant UQ component as Activity~5 (Report~26, also Report~24 below) and additional effort
%under Activity~6 (Reports~12,~21).

For Activity 3, UKAEA produced six milestone reports, namely
\begin{itemize}
\item CD/EXCALIBUR-FMS/0022-M3.1.2 - User frameworks for tokamak multiphysics 
\item CD/EXCALIBUR-FMS/0024-M3.1.3 - User layer design for uncertainty quantification
\item CD/EXCALIBUR-FMS/0023-M3.3.2 - Design patterns specifications and prototypes 
\item CD/EXCALIBUR-FMS/0026-M3.3.3 - Design patterns evaluation
\item CD/EXCALIBUR-FMS/0032-D3.3 - Module Guide
\item CD/EXCALIBUR-FMS/0033-D3.4 - Development Plan
\end{itemize}
and there were three earlier milestone reports from FY2019/20, namely
\begin{itemize}
\item CD/EXCALIBUR-FMS/0014-M3.1.1 - \nep: Report on system requirements
\item CD/EXCALIBUR-FMS/0015-M3.3.1 - \nep: Background information and user requirements for design patterns
\item CD/EXCALIBUR-FMS/0016-M3.5.1 - Benchmarking requirements for \nep\  and available tools
\end{itemize}

These reports are concerned principally with assessing the state-of-the-art in software
design with a particular emphasis on design of scientific software, and of course paying attention to
Exascale applicability. Selected textbooks and the wider literature were examined for the
factors important for successful software
developments. This examination threw up the importance of (1) software frameworks 
described in Report~22 as an integrated set of software artefacts that
collaborate to provide a reusable architecture for a family of related applications,
(2) software layering in Report~24 as a more widely useful  technique than just one enabling `separation of concerns',
and (3) design patterns in Reports~15, 23 and~26 as an approach to reusing and communicating
reliable software structures. 
The importance of building a community and the techniques for
doing so were also described in Report~22.

Report~32 describes how the concept of module or class
should be integrated into a structure of frameworks, layers and design patterns, so that
a large, complex code can be partitioned into manageable segments. The utility of 
the Unified Modeling Language (UML~2) to describe not just software structure,
but also code use in a wider based engineering structure, through
model-based systems engineering (MBSE) was noted.
Report~33 attempts a preliminary synthesis of the material
presented in the previous Activity~3 reports into a plan for the \nep\ software life-cycle,
with focus on the subdivision of the plan into documents expected to be arranged as a web-site.
Report~24 also includes an introduction
to a wide range of uncertainty quantification~(UQ) techniques.
