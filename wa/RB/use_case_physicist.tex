\subsection{Use Cases: Tokamak edge physicist}\label{sec:edge-boundary-tokamak-physicist}

They are early career, and to progress they need to build a professional
reputation by publishing papers, supporting UKAEA's research programmes
and supervising students. They are a competent developer and experienced
HPC user, though they do not gain any credit directly from developing
software.

In their research work, they study different models for the tokamak
edge, and so require code flexibility and a user-friendly DSL to allow
them to rapidly prototype different equation sets. This work would
require quick iterations -- perhaps 5 minute simulations performed on a
desktop. They will also develop their own algorithms and add
infrastructure to the code. While they will do this with an
understanding of performance implications, they would expect to perform
these developments at a higher level that raw performance loops (but at
a lower level than the physics model).

They would expect to contribute their changes back to a community
repository, and also to benefit from changes that other code users have
made. They would be involved in the community -- perhaps raising issues,
making and reviewing git pull requests, answering queries, and having input into
future code releases -- but would not be involved ``project management''
tasks, like maintaining the repository.

They will also value a user-friendly interface and active user community
when it comes to working with their students. In this context it is
valuable to have software that will run at a high level and produce
sensible results without needing to specify the details of the
implementation. This allows the student to learn about physical systems
without simultaneously having to learn the details of numerical
implementations. The active community allows their student to get
support and ask (perhaps trivial) questions without being dependent on
their supervisor.

Finally, in support of experiments, they will need to perform
high-fidelity simulations of tokamaks. These will be highly
computationally expensive, either because they are high-resolution
simulations of specific shots, or because they are parameter scans or UQ
campaigns. The simulations will be long-running, perhaps in the range of
a week to a few months, on whichever HPC system that they have access
to. The software must therefore be performance portable in order to
facilitate high performance on a range of systems. The software also
needs to be robust to numerical instabilities, hardware node failures,
etc, as one may not have the resource allocation to repeat failed runs.
