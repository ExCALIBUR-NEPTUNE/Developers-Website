%This is an attempt at a profile of a certain class of \nep \   user stating 
%what might be reasonable expectations of the code and its interface.
\subsection{Use Cases: Finite Element Background}\label{sec:use-cases-finite-element-background}

I am a user with perhaps some grasp of plasma physics but with a more extensive 
knowledge of finite-element software (I might be an experienced user / 
developer of Nektar++).  My background may be either physics or engineering; I 
may be a new recruit to the \nep \ team and needing to learn the code with a 
view to taking a future role as a \nep \ developer.

I need the interface / DSL to provide access to typical FEM parameters eg.\ 
choice of intra-element basis functions and their polynomial order, continuous 
/ discontinuous Galerkin, choice of numerical flux, stabilization options; also 
whether diffusion and advection terms are explicit or implicit.  In line with 
eg.\ Nektar++ I expect the choice of time-stepper to be largely ``orthogonal'' 
to most details mentioned above (the exception is explicit / implicit choice).  
I would like the option to specify the timestep in terms of the CFL number.
In addition I require control 
over relevant meshing parameters eg.\ element spatial density and approximation 
order of any curvilinear elements.
I would like the DSL to be able to generate a range of regular meshes 
internally (at least for trivial cases eg.\ boxes meshed with quads).

I should like some simple, physically-motivated canonical examples that might
assist with learning plasma physics.

I expect the performance of the code to be at least commensurate with other FEM 
packages eg.\ Nektar++ and to remain so going forward (and obviously must be 
scalable to the latest hardware, which means foreseeably an efficient GPU 
implementation, supporting ideally NVidia, AMD, and Intel Xe / Ponte Vecchio).

I am unused to velocity-space effects.
I would like the particles aspects of the code to be expressible, insofar as is 
possible, in FEM language: the conversion from discrete to continuum should 
ideally not be visible to me eg.\ converting particles to FEM forcing terms.  
Further to this, it would be good if a set of default particle parameters could 
be produced based on FEM parameters, as required (perhaps a reasonable value 
for the number of particles can be derived automatically based on FEM 
resolution).

If there is an issue from PIC compatibility (eg.\ constrained choice of basis 
functions), the DSL should make this clear in an explicit error message, plus 
hopefully advice how to remedy.

%The interface to higher-dimensional kinetic aspects needs careful thought.

