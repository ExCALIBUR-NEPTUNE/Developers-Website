\newsection{FabNEPTUNE}{sec:FabNEPTUNE}

Intended as a general tool for running NEPTUNE workloads, FabNEPTUNE is a plug-in for the FabSim3 Python-based toolkit
 for scientific workflow automation.  The framework generally enables the execution of simulation and analysis workloads
 on HPC via one-liner commands, automation of coupled models, facilities for reproducibility, and tools for ensemble
execution.  The framework offers also integration with the VVUQ toolkit SEAVEAtk. 

The current implementation of FabNEPTUNE is set-up to run simulations of heat transport by fluid convection, in 2D and
 3D, using the incompressible Navier-Stokes solver of the Nektar++ spectral/hp finite element framework.  The test 
problem simulates a fluid-filled rectangular cavity subject (via Dirichlet boundary conditions) to a specified 
horizontal temperature gradient, which leads to the generation of a convective circulation as fluid rises up the hot 
side and descends the cold.  There are two main control parameters represented by the dimensionless Nusselt and 
Prandtl numbers, with the former being a measure of the size of the applied temperature gradient - tuning this 
allows access to conductive, laminar-convective, and turbulent regimes.  This problem is relevant to fusion by 
analogy (\cite{Wi19Stab}) and is also a demonstration of the Nektar++ code.  See the internal reports \cite{y3re61,
y3re62} for numerical studies of the 2D convection problem using Nektar++.

FabNEPTUNE is publicly available at \url{https://github.com/UCL-CCS/FabNEPTUNE}.
