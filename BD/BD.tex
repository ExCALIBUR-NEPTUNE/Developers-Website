% sections 1 to 7 of Hewitt  - Business Design
% Sommerville  - Introduction
\begin{enumerate}
\item  {\bf CAPABILITIES}

The \nep \ package will be capable of efficiently contributing `actionable' results regarding
the first wall to the reactor design process, specifically by speedily modelling the power deposited by the plasma
\begin{itemize}
\item subject to uncertainty quantification~(UQ)
\item using a modular suite of compatible components of minimal necessary complexity 
so as to ensure a workflow for rapid design,
capable when needed of involving the latest high performance computers (HPC)
\end{itemize}
In addition, the package will facilitate  research into edge plasma physics by
\begin{itemize}
\item its ease of use, providing a suitable DSL for `high-level' usage in Python/Julia to enable
intuitive additions to existing models or the incorporation of new models, whereby
equations may be defined using \LaTeX \ as explicit PDEs or as Lagrangians, %, as Lie derivative.
and novel initial conditions imposed and new boundary conditions applied
\item its ease of modification and development, by
providing a set of well-defined objects/classes for tokamak plasma physics
\item ease of incorporation of its component software into other physics packages
\item offering a careful automatic control of numerical error
\end{itemize}

\item  {\bf STRATEGIC FIT} % - to UKAEA mission

{\bf National}: The \exc \ project is of national importance to UK government
(represented initially by the BEIS and subsequently the DESNZ dept.)  to demonstrate
how to produce software that can exploit all the latest, most powerful hardware
for scientific computation.  The Fusion Modelling System~(FMS) is one
of \exc 's principal use cases, for which project \nep \ will explore efficient development of
new software for the Exascale.

%UKAEA researches fusion energy and related technologies, with the aim of positioning
%the UK as a leader in sustainable nuclear energy.
{\bf UKAEA}: The software will facilitate world-leading R\&D into fusion energy by UKAEA physicists and
engineers, underpinned by external collaborators with a wide range of expertise.
Its research contribution will be to improve detailed physical understanding
of the often-turbulent plasma-wall interaction. Development of tokamak reactors will
be facilitated if Exascale machines can be seamlessly integrated into the design workflow.
\item  {\bf BUSINESS DRIVERS} % - why are we doing this?

Optimal power-handling at the wall will be critical for fusion reactors to be able to
deliver sustainable fusion energy to the grid economically, if at all.
%Accurate detailed modelling requires large computing resources,
Rapid exploration of parameter space is important to understand and optimise
reactor designs, yet the existing software available to UK is dated so that it
require years of experience to use well and is not suited to the latest HPC.
Its replacement should remove a major handicap in the race to produce reactor designs.
\item  {\bf ASSUMPTIONS} % regarding available funding

Funding of approx.\ $\pounds$\,5\,M over $5$~years has been made available to UKAEA via the Strategic
Priorities Fund under its \exc \ programme to ready the UK for the Exascale era of computing.
It is expected that further development of the \nep \ software will be funded at a similar if
somewhat smaller level after 2025.
%\item  {\bf (CONSTRAINTS)} % laws and regulations)
\item  {\bf RISKS} %  if resources not available

Funding is via UK government and does not depend on the international situation.
\item  {\bf IMPACTS} % new processes, training needed

The new software should enable much faster iteration in respect of engineering design of the first wall
of tokamak reactors. It should save much time and effort in the modelling of
tokamak edge plasma physics. Generally, it should greatly reduce the training and computer time
needed to obtain results compared to the existing software. There should also be benefits to
UKAEA's wider relationships with the Eurofusion E-TASC programme and with ITER.
\item  {\bf STAKEHOLDERS} % who will win or lose by good or bad outcome

The successful outcome of the project should be a plus for the UK, UKAEA and most employees.
The only losers will be those UKAEA staff and contractors who have devoted often years of
their working lives to the dated software that \nep \ is
designed to replace, and will undergo a loss of status in consequence. However, their physics skills
and understanding will be valuable for guiding the \nep\ development to produce maximum
benefits, hence they should soon recover position within UKAEA.
\item  {\bf GOVERNANCE} % committees

The \nep \ project is overall governed by the UK governance known as PRINCE2~\cite{prince2,prince2wiki}.
which demands oversight by a local committee known as a project board.
Finance is subject to the usual UKAEA procedures and controls.
The \nep \ project is further subject to
reporting to UK Met Office as part of wider \exc \ activities, whence there is a
second layer of PRINCE2 oversight.

The planned technical activities are outlined in the Science Plan~\cite{sciplan},
it and all  major changes and refinements are subject to external refereeing,
following the \exc \ procedures drawn up by UK Met Office consistent with the 
demands of the UK Strategic Priorities Fund~\cite{SPF}.
External procurements similarly follow the \exc \ procedures drawn up by UK Met Office.
\end{enumerate}

%There are
%deficiencies in the existing plasma physics software base for modelling the
%tokamak edge which should be addressed, and there is the question of how best to meet them. There is the issue of
%whether to exploit opensource software, and if so how best to do so. Factors such as
%future-proofing, speed of code execution, the need for uncertainty
%quantification,
%the interaction between different packages for CAD, meshing and physics codes are
%accounted for in the discussion. The avoidance of ``heroic computing" in favour of UQ
%could lead to considerable simplification in choosing a way forward, probably without
%any significant long-term drawbacks.
