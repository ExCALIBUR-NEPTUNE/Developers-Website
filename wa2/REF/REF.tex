% Sommerville  - Glossary
% Smith  - Requirements  - Reference Material
% Smith  - Requirements  - Specific System Description  - Problem Description  - Terminology and Definitions
% Hewitt  - Infrastructure Design  - Standards and guidelines and conventions
% Hewitt  -  Acronyms and Symbols
\newsectionnobreak{Conventions for Report Writing}{sec:conv}
The project has strict standards in respect of conventions even in reports
that contain no code, ultimately to support the `write once, use many times' concept,
so that text may be cut, edited and pasted into source code, or indeed used safely to define
variable names, constraints on their values and physical dimensions.
Thus, since many compilers support only ASCII characters, only ASCII is allowed
in reports. Similarly, very long lines (typically arising from MS Windows wordprocessing)
should be split at the ends of sentences, and preferably so that maximum line length
is $120$~characters. A number of tools are available in \T{tex/importools}.
Contributors may use their style of choice for both reports (and code) provided
they follow systems that allow for automatic conversion to the conventions
expected in the \nep \ repositories, and are prepared to help construct gitlab runners and
github hooks to this end. However, it will be generally found easier to use \LaTeX \ and
bibtex as indicated below.

For shorter documents it is acceptable to use Markdown in the `dialect' defined by
\F{PANDOC}, see Annex~A of~\cite{y3re314}. Resulting .md files frequently convert
straightforwardly to \LaTeX \ format, using the md2tex.bash script from \T{tex/importools}.


\subsection{General textual issues}\label{sec:REF_text}
It is helpful to use \LaTeX \ newcommand for certain keywords where they have a special
format or because the terminology is not fully established. Although some do not, most
people do find variations in use of spelling, punctuation, capitalisation etc.\ to be irritating,
and so the following conventions will be enforced where possible: 
\begin{itemize}
\item use of \LaTeX \ newcommand for , viz. \verb-\-papp \ and  \verb-\-exc \ instead of explicit
proxyapp or mini-app and \exc \  respectively
\item punctuation, viz.\ eg.\verb-\- rather than e.g.
\item hyphenation, generally to be avoided because \LaTeX \ may use it to break lines, viz.\ finite element rather than finite-element (or simply FE), opensource rather than open source, major exception is abbreviation of dimensions (below).
\item capitalisation, eg.\ Exascale rather than exascale
\item abbreviation of dimensions, 2-D and 3-D preferred to 2D and 3D or 2d and 3d
\item spelling, UK English preferred, and  '-ise' etc.\ preferred to '-ize'
\item no spaces between authors' initials in bibtex files (see also \Sec{REF_citations})
\item consistent usage of acronyms, employing the table of \Sec{acro}
\item consistent usage of mathematical symbols, employing the table of \Sec{symbol}
\end{itemize}

\subsection{Citations}\label{sec:REF_citations}
Regarding citations, those for unpublished reports MUST include a link to a
website, such as arXiv, and it would be helpful if links to other open-access
material were also given.
Use of the following conventions for constructing the keys of citations should help avoid clashes:
\begin{enumerate}
\item For published papers, use where possible an $8$-character alphanumeric for published papers,
consisting of the first two letters of the first author's name, the last two digits
of the year in the Gregorian calendar, and the first $4$~letters of the first \emph{significant}
word (ie.\ not `The' or~`On') of the title, preserving capitalisations.
Thus the paper~\cite{Ba13What} by Bangerth and Heister, published in 2013 and
entitled `What makes computational open source software libraries successful?',
has key~`Ba13What'.
\item Books and theses should be keyed with the full second-name(s) of the author(s) strung together
without capitalisation, up to a limit of approx.~$30$ characters, using `etal'
to indicate any omitted author-names. As an example, the book by Rouson, Xia
and~Xu~\cite{rousonxiaxu} has key~`rousonxiaxu'.
\item If there are duplicates in different files, then preface
each key with the name of the .bib file it is in, for other duplicates within a file,
add `2', `3' etc.\ to the end of the key.
\end{enumerate}

\subsection{Software Compatibility}\label{sec:REF_compat}
% to TS % Latest C++, C++20 if possible including modules. (like Fortran-95 only Import instead of use), and
% to TS %             Concepts - generalised types of single variable.
% to TS % Follow Stroustrop~\cite{stroustroptour}.
% to TS % Use of clang\_tidy.

When discussing software in the text, the following conventions in \LaTeX\  are proposed:
\begin{itemize}
\item \F{Small capitals} denote a package name, use \{\verb-\-textsc or abbreviation \verb-\-F\{
\item \I{Italics} denote a program name, use \{\verb-\-textit or abbreviation \verb-\-I\{
\item \T{Fixed width font} denotes any code name or fragment which is not otherwise obviously source code,
use \{\verb-\-texttt or abbreviation \verb-\-T\{
\end{itemize}
There is no need for special fonts if the object is identified by a
suitable suffix, thus ``\_m" for a module containing executable code,
``\_h" or ``.h" for an object description or namespace code, ``.cpp" for name of file containing
C++ source, ``.exe" for an executable, etc. Similarly file suffices that imply a particular
format or software for their interpretation, may simply be written with a leading stop,
eg.\ ``.html" and ``.exe".

In order to ensure smooth transliteration from mathematical symbols to the names
of the software variables, \Tab{twoclatex} lists the recommended two-character
equivalents for \LaTeX \ symbols used in the definition of mathematical symbols.
\begin{table}[tbph]
\begin{center}
\caption{\textbf{\textsf{TWO CHARACTER EQUIVALENTS}} of \LaTeX \ symbols and commands.  \label{tab:twoclatex} }
\begin{tabular}{||p{1.5cm}|p{4cm}||p{1.5cm}|p{4cm}||p{1.5cm}|p{4cm}||}
\hline
aa & $A$ & al & $\alpha$ & ar & $\rightarrow$ \\
as & $*$ & bb & $B$ & be & $\beta$ \\
bl & $[$ & br & $]$ & cc & $C$ \\
ch & $\chi$ & ci & \verb-^- & dd & $D$ \\
de & $\delta$ & dl & $\Delta$ & dq & " \\
ds & \verb-\-ddot & dv & $/$ & ee & $E$ \\
el & $\ell$ &  &  & &  \\
ep & $\epsilon$ & et & $\eta$ & ff & $F$ \\
ga & $\gamma$ & gg & $G$ & gm & $\Gamma$ \\
gt & $>$ & hh & $H$ & ii & $I$ \\
in & $\infty$ & & & & \\
it & $\iota$ & jj & $J$ & ka & $\kappa$ \\
kk & $K$ & la & $\lambda$ & ll & $L$ \\
lm & $\Lambda$ & lt & $<$ & me & $\omega$ \\
mg & $\Omega$ & ml & $\times$ & mm & $M$ \\
mn & $-$ & mu & $\mu$ & n1 & $n_1$ etc. \\
nn & $N$ & nu & $\nu$ & o2 & \verb-\-boldmath \\
o3 & \verb-\-mathcal & o4 & \verb-\-mathsf & o5 & \verb-\-mathtt \\
o6 & \verb-\-mathbb & o7 & \verb-\-mathfrak & \\
oe &  suffix  & ol &  preceding superfix  & on &  above  \\
or &  superfix  & os &  underneath  & ow &  prefix  \\
pa & $\|$ & pe & $\perp$ & pf & $\Phi$ \\
ph & $\phi$ & pi & $\pi$ & pj & $\Psi$ \\
pl & \{ & pp & $P$ & pr & \} \\
ps & $\psi$ & pt & $\partial$ & pu & $+$ \\
py & $\Pi$ & qq & $Q$ & rh & $\rho$ \\
rr & $R$ & sg & $\Sigma$ & si & $\sigma$ \\
sq & $'$ & ss & $S$ & st & $.$ \\
ta & $\tau$ & te & $\Theta$ & th & $\theta$ \\
ti & $~$ & tt & $T$ & un & \_ \\
up & $\upsilon$ && && \\
us & $\Upsilon$ & uu & $U$ & vb & $|$ \\
ve & $\varepsilon$ & vf & $\varphi$ & vp & $\varpi$ \\
vr & $\varrho$ & vs & $\varsigma$ & vt & $\vartheta$ \\
vv & $V$ & ww & $W$ & xx & $X$ \\
yy & $Y$ & ze & $\zeta$ & zz & $Z$ \\
\hline
\end{tabular}
See \Sec{two-character-variables} for a guide explaining the reasons for the above choices.
\end{center}
\end{table}



\clearpage
\newsection{Acronyms}{sec:acro}
\begin{longtable}{|p{4.0cm}|p{12.0cm}|}
\caption{\textbf{\textsf{TABLE OF ACRONYMS}} Nearly all the acronyms refer to technical
terms. A debt is acknowledged to the book by Brunton and Kutz~\cite{bruntonkutz}.
} \\
\hline
ACM & Association for Computing Machinery\\
ADC & Analogue to digital converter\\
ADM  & Alternating directions method \\
AIC  & Akaike information criterion \\
ALM  & Augmented Lagrange multiplier \\
AMR & Adaptive mesh refinement\\
AMReX & Software framework for block-structured AMR\\
ANL & Argonne National Laboratory \\
ANN & Artificial Neural Network \\
ANOVA & Analysis of Variance \\
API & Application Programming Interface \\
ARMA  & Autoregressive moving average \\
ARMAX  & Autoregressive moving average with exogenous input \\
ASQ & Adaptive sparse quadrature \\
ATS & Advanced Terrestrial Simulator, previously Arctic Terrestrial Simulator \\
BC & Boundary Condition\\
BEIS &  (UK government) Department for Business, Energy and Industrial Strategy \\
BG/L & IBM Blue Gene / L supercomputer platform\\
BIC  & Bayesian information criterion \\
BOUT++ & Tokamak edge plasma modelling framework \url{https://boutproject.github.io/} \\
BPOD  & Balanced proper orthogonal decomposition \\
BSD  & Opensource software licence \\
%CabanaMD & \\
CAD & Computer-Aided Design, geometry including NURBS, usually ``CAD database" implied \\
CCA  & Canonical correlation analysis \\
CCFE & Culham Centre for Fusion Energy \\
CEA  & The French Alternative Energies and Atomic Energy Commission \\
CESM & Community Earth System Model\\
CFD  & Computational fluid dynamics \\
CI & Continuous integration\\
CLI & Command Line Interface \\
CNN  & Convolutional neural network \\
COGENT & LLNL continuum plasma simulation code\\
COMPAT & Computing patterns for multiscale HPC (project)\\
CoSaMP  & Compressive sampling matching pursuit \\
COSMO & Framework for regional weather prediction in Europe \\
COSSAN & UQ and risk analysis package (Uni. Liverpool)\\
CPP & C plus plus programming language\\
CPU & Central Processing Unit \\
CRUD & Create, Read, Update, Delete \\
CS & Compressed sensing \\
CSE & Computational science and engineering\\
CSG & Constructive Solid Geometry \\
CSMP & Computer science, mathematics, and physics\\
CTO & Chief Technology Officer \\
CUDA & Compute Unified Device Architecture \\
CWIPI & Coupling with interpolation parallel interface (coupling library)\\
CWT  & Continuous wavelet transform \\
DA & Data Assimilation\\
DAG & Direct Acyclic Graph \\
DAKOTA & UQ and optimization package (Sandia)\\
DCT  & Discrete cosine transform \\
DDA & Digital Differential Analyser\\
DDD & Document-Driven Design \\
DE & Differential equation \\
DEIM & Discrete Empirical Interpolation Method \\
DFT & Discrete Fourier Transform \\
DiMDc  & Dynamic mode decomposition with control \\
DL  & Deep learning \\
DMD  & Dynamic mode decomposition \\
DMDc  & Dynamic mode decomposition with control \\
DNS  & Direct numerical simulation \\
DOE & Department of Energy \\
DOI & Digital Object Identifier \\
DPC++ & Data Parallel C++, Intel compiler for C++ with SYCL extension \\
DRAM & Delayed Rejection Adaptive Metropolis \\
DSL & Domain-Specific Language \\
DWT  & Discrete wavelet transform \\
ECOG  & Electrocorticography \\
ECP & Exascale Computing Project \\
ECP-copa & Co-design centre for particle applications (part of ECP)\\
eDMD  & Extended DMD \\
EIM  & Empirical interpolation method \\
EIRENE & name of neutral package  \\
EM  & Expectation maximization \\
EOF  & Empirical orthogonal functions \\
ERA  & Eigensystem realization algorithm \\
ESC  & Extremum-seeking control \\
ESI & name of software company \url{https://www.esi-group.com/}  \\
ESMF & Earth System Modeling Framework \\
E-TASC & EUROfusion Theory and Advanced Simulation Coordination \\
ETS & European Transport Simulator\\
EU & European Union \\
FCI & Flux-Coordinate Independent (method) \\
FELTOR & name of edge code \\
FEM & Finite Element Method \\
FEniCS & name of PDE software project \url{https://fenicsproject.org} \\
FFT & Fast Fourier Transform \\
FFTW & Fastest Fourier Transform in the West (library) \\
FLASH & name of Multiscale physics code \\
GA & General Atomics \\
GBS & Global Braginskii Solver (software)\\
GCR & Generalied Collisional Radiative (framework) \\
GDB & Global Drift-Ballooning \\
GDB & GNU debugger \\
GDPR & General Data Protection Regulation \\
GENE & name of gyrokinetic code \\
GMM  & Gaussian mixture model \\
GMRES & Generalized Minimal Residual method \\
GNU & GNU's Not Unix! \\
GP & Gaussian Process \\
gPC & Generalised polynomial chaos (Xiu and Karniadakis \url{https://doi.org/10.1016/S0021-9991(03)00092-5} \\
GPU & Graphics Processing Unit \\
GRILLIX & name of 3D turbulence code based on the flux-coordinate independent approach \\
GSA & Global sensitivity analysis \\
GUI & Graphical User Interface \\
HAGIS & HAmiltonian GuIding centre System\\
HAVOK  & Hankel alternative view of Koopman \\
HDF5 & Hierarchical Data Format (version 5) \\
HDS & Hierarchical Data Structure \\
HLA & High Level Architecture\\
HPC & High Performance Computing \\
HTC & High Throughput Computing \\
IBM & International Business Machines Corp., but really known as IBM \\
IC & Initial Condition \\
ICA  & Independent component analysis \\
ICON & ICOsahedral Nonhydrostatic, the global numerical weather prediction model of the German weather service \\
IEEE & Institute of Electrical and Electronics Engineers \\
IETF & Internet Engineering Taskforce \\
IMAS & Integrated Modelling \& Analysis Suite, promoted by ITER \\
IMEX & Implicit-Explicit Methods \\
IO & Input/Output \\
ITER & name of International Thermonuclear Experimental Reactor \\
ITG & Ion Temperature Gradient \\
ITM & Ion Tearing Mode \\
ITPA & International Tokamak Physics Activity (ITER research programme)\\
JET & Joint European Torus \\
JIT & Just In Time \\
JL  & JohnsonLindensfrauss \\
JOREK & name of nonlinear MHD code\\
JSON & JavaScript Object Notation \\
KL  & Kullback Leibler \\
KLT  & Karhunen-Loeve transform \\
LAD  & Least absolute deviations \\
LAMMPS & Large-scale Atomic/Molecular Massively Parallel Simulator\\
LANL & Los Alamos National Laboratory \\
LASSO & Least Absolute Shrinkage and Selection Operator \\
LCFS & Last Closed Flux Surface \\
LDA  & Linear discriminant analysis \\
LGPL & GNU Lesser General Public License \\
LHSamp & Latin Hypercube Sampling\\
LLNL & Lawrence Livermore National Laboratory \\
LOO & Leave One Out  \\
LQE  & Linear quadratic estimator \\
LQG  & Linear quadratic Gaussian controller \\
LQR  & Linear quadratic regulator \\
LTI  & Linear time invariant system \\
MAP & Maximium A Posteriori \\
MBSE & Model-based systems engineering \\
MC & Monte-Carlo (methods) \\
MCMC & Markov chain Monte-Carlo \\
MCT &  Model Coupling Toolkit \\ % MCT is mentioned here and in a comment?
MD & Molecular Dynamics \\
MECE & Mutually exclusive and collectively exhaustive \\
MF & Multi-fidelity, Matrix-free \\
MFMC & Multi-fidelity Monte-Carlo \\
MHD & Magnetohydrodynamics \\
MIMC & Multi-Index Monte-Carlo \\
MIMO  & Multiple input, multiple output \\
MIS & Module Interface Specification \\
MIT & Massachusetts Institute of Technology \\
MIT licence& Opensource software licence~\cite{MITlicense} \\
ML & Machine Learning \\
MLC  & Machine learning control \\
MLMC & Multi-Level Monte-Carlo \\
MLMF & Multi-Level Multi-Fidelity\\
MMF & Multiscale Modeling Framework\\
MMS & Method of Manufactured Solutions \\
MOOSE & Multiphysics Object Oriented Simulation Environment\\
MOR & Model Order Reduction\\
%Most  & Common Acronyms \\
MPE  & Missing point estimation \\
MPI & Message Passing Interface \\
mrDMD  & Multi-resolution dynamic mode decomposition \\
MSSC & Materials Science and Scientific Computing\\
MUMPS & MUltifrontal Massively Parallel Sparse direct Solver\\
MUSCLE~3 & Multiscale Coupling Library and Environment version 3\\
NAG & Numerical Algorithms Group \\
NARMAX  & Nonlinear autoregressive model with exogenous inputs \\
NEMO & Nucleus for European Modelling of the Ocean\\
NEPTUNE & Neutrals and Plasma Turbulence Numerics for the Exascale \\
NetCDF  &  Network Common Data Form \\
NLS  & Nonlinear Schroedinger equation \\
NROY & Not ruled out yet \\
NUCODE & Software: SMARDDA/NUCODE for Neutral Beam Duct Calculations\\
NURBS & NonUniform Rational B-Spline \\
OASIS &  Ocean Atmosphere Sea Ice Soil\\
OASIS4 &  Ocean Atmosphere Sea Ice Soil version 4\\
ODE & Ordinary Differential Equation \\
OKID  & Observer Kalman filter identification \\
OLYMPUS & OLYMPUS Programming System\\
OMFIT & One Modeling Framework for Integrated Tasks\\
OneAPI & A Unified, Standards-Based Programming Model, \url{https://software.intel.com/en-us/oneapi}\\
OP2 & API with associated libraries and preprocessors for performance-portable parallel computations on unstructured meshes \url{https://github.com/OP-DSL/OP2-Common}\\
OpenMP & Open Multi-Processing \\
OU & Oxford University \\
OUU & Optimisation under uncertainty \\
PASTIX & Parallel Sparse matriX package\\
PBH  & PopovBelevitchHautus test \\
PC & Polynomial chaos \\
PCA  & Principal components analysis \\
PCE & Polynomial chaos expansion \\
PCP  & Principal component pursuit \\
PDE & Partial Differential Equation \\
PDE-FIND  & Partial differential equation functional identification of nonlinear dynamics \\
PDF  & Probability distribution function \\
PETSc & Portable Extensible Toolkit for Scientific Computation \url{https://www.mcs.anl.gov/petsc/}\\
PFC & Plasma Facing Component\\
PGD & Proper Generalised Decomposition \\
PIC & Particle-In-Cell\\
PICPIF & Particle-In-Cell-Particle-In-Fourier\\
PID  & Proportional-integral-derivative control \\
PIV  & Particle image velocimetry \\
POD & Proper Orthogonal Decomposition \\
POOMA & Parallel Object-Oriented Methods and Applications \\
PP20 & SIAM Conference on Parallel Processing for Scientific Computing 2020\\
PPMD & Performance-Portable Framework For Atomistic Simulations \\
PR & git Pull Request \\
PSyclone & PSyclone is a code generation system that generates appropriate code for the PSyKAl code structure developed in the GungHo project.  \url{https://github.com/stfc/PSyclone}\\
PyOP2 & Framework for performance-portable parallel computations on unstructured meshes \url{http://op2.github.com/PyOP2}\\
QA & Quality Assurance\\
QCG  & Quality in Cloud and Grid, see QCG Pilot Job\\
QMC & Quasi-Monte-Carlo \\
QoI & Quantity of interest \\
QoS & Quality of Service \\
RAID & Risks, Assumptions, Issues, Dependencies \\
RAJA & RAJA Performance Portability Layer (C++) \url{https://github.com/LLNL/RAJA} \\
REST & Representational State Transfer (Resources as simple CRUD objects) \\
RIP  & Restricted isometry property \\
RKF23 & Runge-Kutta-Fehlberg (\emph{aka} Embedded Runge-Kutta), $23$ denotes orders of scheme\\
RKHS  & Reproducing kernel Hilbert space \\
RMS & Root-mean-square \\
RNG & Random Number Generator\\
RNN  & Recurrent neural network \\
RO & Responsible Officer  \\
ROM & Reduced-Order Model \\
RPCA  & Robust principal components analysis \\
rSVD  & Randomized SVD \\
SAMRAI & Structured Adaptive Mesh Refinement Application Infrastructure\\
SD1D & name of 1-D edge code \\
SDLC & Software Development Life Cycle \\
SGD  & Stochastic gradient descent \\
SIAM & Society for Industrial and Applied Mathematics \\
SINDy  & Sparse identification of nonlinear dynamics \\
SISO  & Single input, single output \\
SLA & Service-level Agreement \\
SLE & Software Language Extensions \\
SLE & System Level Engineering \\
SLEPc & name of Scalable Library for Eigenvalue Problem Computations \\
SLSQT & Sequential Least-Squares' Thresholding \\
SMARDDA & name of Ray-tracing algorithm, hybrid of SMART and DDA \\
SMART & name of Ray-tracing algorithm based on use of octree\\
SMITER & SMARDDA modules with ITER interface \\
SNOWPAC & Stochastic Nonlinear Optimisation with Path-Augmented Constraints (software package) \\
SOL & Scrape-Off Layer \\
SOLEDGE & name of edge modelling code \\
SOLPS & name of edge modelling code combines B2 and EIRENE\\
SPH & Smoothed Particle Hydrodynamics \\
SRC  & Sparse representation for classification \\
SRO & Senior Responsible Owner role in UK  government project delivery \\
SRS & Software Requirements Specification \\
SSA  & Singular spectrum analysis \\
SSD & Scientific Software Development \\
StarPU & Runtime system supporting heterogeneous multicore architectures \url{http://starpu.gforge.inria.fr/doc/html/} \\
STARWALL & name of vacuum field code \\
STFT  & Short time Fourier transform \\
STIX & Scientific And Technical Information eXchange \\
STLS  & Sequential thresholded least-squares \\
STORM & Scrape-off layer Transport ORiented Module \\
STRUMPACK & STRUctured Matrix PACKage \\
SUNDIALS & name of ODE package \\
SVD & Singular Value Decomposition \\
SVM & Support Vector Machine \\
SYCL & C++-single-source heterogeneous programming for acceleration offload, \url{https://www.khronos.org/sycl/} \\
SysML & Systems Modeling Language  \\
TAE & Toroidal Alfven Eigenmode \\
TDD & Test Driven Development \\
TICA  & Time-lagged independent component analysis \\
TM & TradeMark\\
TOKAM & name of set of edge modelling codes\\
TOKAM3X & name of Edge modelling software \\
TOMS & Transactions on Mathematical Software \\
TORPEX & TORoidal Plasma Experiment \\
Trilinos & Object-oriented software framework for the solution of large-scale, complex multi-physics engineering and scientific problems \url{https://trilinos.github.io/}\\
TRIMEG & TRIangular MEsh based Gyrokinetic code\\
TSVV & Theory, Simulation, Validation and Verification, tasks of the E-TASC programme of Eurofusion \\
TUM & Technical University Munich\\
UK & United Kingdom \\
UKAEA & United Kingdom Atomic Energy Authority \\
UKRI & United Kingdom Research and Innovation, a non-departmental public body encompassing the research councils and Innovate UK \\
UML & Unified Modelling Language\\
UQ & Uncertainty quantification \\
US & United States \\
USA & United States of America\\
UTF-8 & Unicode Transformation Format (Unicode denotes Universal Coded Character Set) \\
UUID & Universally Unique IDentifier is a 128-bit label used for information in computer systems\\
VAC  & Variational approach of conformation dynamics \\
VDE & Vertical Dispacement Event\\
VECMA & Verified Exascale Computing for Multiscale Applications \\
VECMAtk & VECMA toolkit \\
VORPAL & name of Electromagnetic Particle-in-Cell code\\
VSVO & variable stepsize, variable order solver of differential equation \\
VVUQ & Verification, Validation and Uncertainty Quantification \\
XGC1 & name of Particle-based gyrokinetic code\\
XML &  eXtensible Markup Language\\
XMSF & eXtensible Modeling and Simulation Framework \\ 
XPN & \exc \  Project \nep \ \\ 
\hline
\end{longtable}


\clearpage
\newsection{Symbols}{sec:symbol}
\begin{longtable}{|p{3.0cm}|p{10.0cm}|p{3.0cm}|}
\caption{\textbf{\textsf{TABLE OF MATHEMATICAL SYMBOLS}}
If no units are given, then quantity is dimensionless, or if the units are given as~$?$,
then the dimensions depend on context.
Generally, the usage of symbols tries to follow that from the Plasma Formulary~\cite{NRLpf07},
in SI units, with temperatures specified as~$kT$ which returns~$J$. The Formulary also
give the fundamental dimensions of the SI units, which should enable checking of dimensional consistency of
equations, eg.\ magnetic field induction is in Tesla~($T$) whence the fundamental
dimension expression gives~$T=kg s^{-1} C^{-1}$.
Note that the symbols are sorted by font as well as alphabet, so that boldface symbols
appear immediately after~`b' (backslashes ignored). The main source for the symbols is
the Equations document~\cite{pappeqs8}, also included are those listed as used in the text by
Karniadakis and Sherwin~\cite{karniadakissherwin}, prefaced by (K+S), plus symbols used in the
report~\cite{y2re313}. \label{tab:symbols}
} \\
\hline
\textbf{\textsf{Symbol}} & \textbf{\textsf{Description}}  & \textbf{\textsf{Units}} \\
\hline
$a$ & minor radius of the torus (horizontal) & $m$ \\
$a_{ij}$ & coefficient of matrix~$A$ &  \\
$A$ & atomic mass of ion &  \\
$A_i$ & atomic mass of ion & \\
$A_\alpha$ & atomic mass of ion species $\alpha$ & \\
$[a,b]$ & arbitrary finite interval  & \\
$\alpha$ & as suffix is species label or index &  \\
$\alpha_n$ & perturbation amplitude &  \\
$\alpha^{Z_p\rightarrow Z}$ & partial dielectronic recombination rate coefficient  & $m^3 s^{-1}$  \\
$\alpha^{Z\rightarrow Z_m}$ & partial dielectronic recombination rate coefficient  & $m^3 s^{-1}$  \\
$b$ & minor radius of the torus (vertical) & $m$ \\
$B_0$ & used to make $B$ dimensionless  &  $T$  \\
$B_s$ & characteristic magnetic field used to make ${\bf B}$ dimensionless  &  $T$  \\
$\bar{N}_Z$ & average number of charge states  &   \\
$B=|{\bf B}|$ & amplitude of the imposed magnetic field  &  $T$  \\
$B_T$ & amplitude of the imposed toroidal magnetic field  &  $T$ \\
$\beta$ & as suffix is species label & \\
$\beta$ & (Glossary) Ratio of plasma pressure to pressure in magnetic field &  \\
${\bf a} = d^2 {\bf x}/dt^2$ & acceleration experienced by a particle  &  $m^2 s^{-1}$ \\
${\bf A}({\bf x},t)$ & magnetic vector potential   &  $T m$ \\
${\bf B}({\bf x},t)$ & magnetic field  &  $T$ \\
${\bf b}$ & unit vector giving the direction of the magnetic field & \\
${\bf E}({\bf x},t)$ & electric field  &  $V m^{-1}$ \\
$E_s$ & characteristic electric field used to make ${\bf E}$ dimensionless  &  $V m^{-1}$ \\
${\bf E}^{+}$ & modified electric field  &  $m^{-2}$ \\
${\bf F}$ & force vector  &  $N$ \\
${\bf u}_\wedge$ & pseudo / thermal velocity component in flux surface normal to field direction  &  $m s^{-1}$ \\
${\bf v}$ & generic velocity  &  $m s^{-1}$ \\
${\bf v}_\alpha$ & velocity of species~$\alpha$  &  $m s^{-1}$ \\
${\bf v}_{\|}$ & fluid velocity directed along fieldline  &  $m s^{-1}$ \\
${\bf v}_\perp$ & fluid velocity component normal to flux surface  &  $m s^{-1}$ \\
${\bf v}_\wedge$ & fluid velocity component in flux surface normal to field direction  &  $m s^{-1}$ \\
${\bf v}_0$ & initial fluid velocity  &  $m s^{-1}$ \\
${\bf v}_{cx}$ & `charge exchange' perpendicular fluid velocity component  &  $m s^{-1}$ \\
${\bf v}_{E \times B}$ & `E cross B' perpendicular fluid velocity component  &  $m s^{-1}$ \\
${\bf v}_e$ & velocity of the electrons  &  $m s^{-1}$ \\
${\bf v}_i$ & velocity of the ion species  &  $m s^{-1}$ \\
${\bf v}_{e \nabla B}$ & `grad B' perpendicular fluid velocity component for electrons  &  $m s^{-1}$ \\
${\bf v}_{i \nabla B}$ & `grad B' perpendicular fluid velocity component for ions  &  $m s^{-1}$ \\
${\bf v}_\textrm{diff}$ & `diffusive' perpendicular fluid velocity component  &  $m s^{-1}$ \\
${\bf x}=\left(x_1,x_2,\dots,x_d\right)$ & is a $d$-dimensional vector  & \\
${\bf x}$ & position  &  $m$ \\
$b_n$ & `b-factors'~ref~\cite[slide 21]{omullane} & \\
$\boldsymbol{\xi}(\theta)$ & multi-dimensional random variable with a specific probability distribution as a function of the random parameter~$0\leq\theta\leq 1$ & \\
$\boldsymbol{B}$ &  (K+S) Basis matrix & \\
$\boldsymbol{D}_{\xi}$ &  (K+S) Elemental derivative matrix with respect to $\xi$ & \\
$\boldsymbol{f}^e$ &  (K+S) Force vector of the $e$th element & \\
$\boldsymbol{H}$ &  (K+S) Helmholtz matrix ($=\mathcal{A}^T \underline{\boldsymbol{H}^e} \mathcal{A}$)) & \\
$\boldsymbol{H}^e$ &  (K+S) Elemental Helmholtz matrix & \\
$\boldsymbol{L}$ &  (K+S) Laplacian matrix ($=\mathcal{A}^T \underline{\boldsymbol{L}^e} \mathcal{A}$)) & \\
$\boldsymbol{\Lambda}(u)$ &  (K+S) Diagonal matrix of $u(\xi_i, \xi_2)$ evaluated at quadrature points & \\
$\boldsymbol{L}^e$ &  (K+S) Elemental Laplacian matrix & \\
$\boldsymbol{M}$ &  (K+S) Mass matrix ($=\mathcal{A}^T \underline{\boldsymbol{M}^e} \mathcal{A}$) & \\
$\boldsymbol{\mathcal{A}}^T$ &  (K+S) Matrix global assembly & \\
$\boldsymbol{M}^e$ &  (K+S) Elemental mass matrix & \\
$\boldsymbol{n}$ &  (K+S) Unit outward normal & \\
$\boldsymbol{\omega}$ &  (K+S and plasma models) Vorticity & $s^{-1}$ or $C m^{-3}$\\
$\boldsymbol{u}^e$ &  (K+S) Vector containing function evaluated at quadrature points & \\
$\boldsymbol{W}$ &  (K+S) Diagonal weight / Jacobian matrix & \\
$\boldsymbol{\xi}(\theta)$ & multi-dimensional random variable with a specific probability distribution as a function of the random parameter~$0\leq\theta\leq 1$ & \\
$B_p$ & amplitude of the poloidal magnetic field  &  $T$  \\
$C_0=\sqrt{K_{MA} T_0}$ & used to make velocities dimensionless  &  $m s^{-1}$ \\
$\cap$ &  (Sets) Set intersection & \\
$\chi$ &  (K+S) Space of trial solutions & \\
$\chi^{\delta}$ &  (K+S) Finite-dimensional space of trial solutions & \\
$\chi_i(\xi)$ &  (FE Basis) Local Cartesian to global coordinate mapping & \\
$c_p$ & specific heat at constant pressure  &  $J kg^{-1} K^{-1}$ \\
$c_s = \sqrt{\frac{kT_i + {Z_i} kT_e}{{m_i}}}$ & approx. plasma acoustic speed  &  $m s^{-1}$ \\
$c_s = \sqrt{\frac{p}{\rho_m}}$ & plasma acoustic speed  &  $m s^{-1}$ \\
$c_{se} = \sqrt{\frac{kT_e}{{m_e}}}$ & acoustic speed of electrons  &  $m s^{-1}$ \\
$c_{si} = \sqrt{\frac{kT_i}{{m_i}}}$ & acoustic speed of ions  &  $m s^{-1}$ \\
$C_S$ & sound speed coefficient in radiation equation  &  $m s^{-1}$ \\
$\cup$ &  (Sets) Set union & \\
$C(x_i, x_j)$ & covariance of random variables $x_i$, $x_j$  & \\
$d$ & number of dimensions over which the integral is performed  & \\
$\delta p_i$ & stress tensor  & $N m^{-2}$ \\
$\delta$ & Kronecker delta & \\
$\delta_D$ & Dirac delta function & \\
$\delta_e$ & energy flux factor at boundary of the electrons  & \\
$\delta=\frac{1}{2}(\delta_e+\delta_i)$ & energy flux factor at boundary of `mean' species  & \\
$\delta_i$ & energy flux factor at boundary of the ion species  & \\
$\delta_\alpha$ & (Glossary) Magnetisation  parameter, species~$\alpha$  gyroradius normalised to~$L$ & \\
$\delta(x)$ & Dirac delta function of continuous real variable $x$  & \\
$D$ & spatial dimensionality of problem &   \\
$D_A$ & diffusion coefficient for plasma charges in a background of neutrals  &  $m^2 s^{-1}$ \\
$D_e$ & diffusion coefficient for electrons, eg.\  in a background of neutrals  &  $m^2 s^{-1}$ \\
$D_{fv\alpha}$ & scale dissipation in  equation for evolution of species velocity ${\bf v}_\alpha$ & \\
$D_n$ & neutral diffusion coefficient  &  $m^2 s^{-1}$ \\
$D_{fp\alpha}$ & scale dissipation in  equation for evolution of species pressure/energy $p_\alpha$ & \\
$D_i$ & diffusion coefficient for ions, eg.\ in a background of neutrals  &  $m^2 s^{-1}$ \\
$|e|$ & absolute value of the charge on the electron & $C$ \\
$e$ &  (K+S) Finite element number $1 \leq e \leq N_{el}$ & \\
$e_{ijk}$ & weighted integral of triple products of $\Psi_i$ of the ion species & \\
$\emptyset$ &  (Sets) Empty set & \\
$\epsilon_0$ & permittivity of free space  &  $F m^{-1}$ \\
$\epsilon_r=t_s/t_0$ & scale factor for transient term  & \\
$\eta_1, \eta_2, \eta_3$ &  (FE Basis) Local collapsed Cartesian coordinates & \\
$\eta_B$ & plasma resistivity after Braginskii  &  $\Omega m$ \\
$\eta_d=\eta_B/\mu_0$ & plasma resistivity, as diffusivity  &  $m^2 s^{-1}$ \\
$\eta_{e{\sf n}}$ & contribution to plasma resistivity, as diffusivity, from electron-neutral interactions  &  $m^2 s^{-1}$ \\
$\eta_{e{\sf n}\|}$ & contribution to plasma parallel resistivity, as diffusivity, from electron-neutral interactions  &  $m^2 s^{-1}$ \\
$\eta_{i{\sf n}}$ & contribution to plasma resistivity, as diffusivity, from ion-neutral interactions  &  $m^2 s^{-1}$ \\
$\eta_{i{\sf n}\|}$ & contribution to plasma parallel resistivity, as diffusivity, from ion-neutral interactions  &  $m^2 s^{-1}$ \\
$f_0$ & constant in the expansion of $f\left(x_1,\ldots,x_d\right)$  & \\
$f_0$ & initial distribution function of the electrons & $m^{-6} s^3$ \\
$f_\alpha$ & distribution function of species~$\alpha$ & $m^{-6} s^3$ \\
$f_e$ & distribution function of the electrons & $m^{-6} s^3$ \\
$f_i$ & distribution function of the ion species & $m^{-6} s^3$ \\
$f_{ij}(x_i,x_j)$ & coefficient in the expansion of $f\left(x_1,\ldots,x_d\right)$  & \\
$f_{ce}= \frac{\omega_{ce}}{2\pi}$ & electron cyclotron frequency & $s^{-1}$ \\
$f_{ci}= \frac{\omega_{ci}}{2\pi}$ & ion cyclotron frequency & $s^{-1}$ \\
$f_{pe}= \frac{\omega_{pe}}{2\pi}$ & electron plasma frequency & $s^{-1}$ \\
$f_{pi}= \frac{\omega_{pi}}{2\pi}$ & ion plasma frequency & $s^{-1}$ \\
$f_i(x_i)$ & coefficient in the expansion of $f\left(x_1,\ldots,x_d\right)$  & \\
$f\left(x_1,\ldots,x_d\right)$ & joint probability distribution  & \\
$f^\mathcal{E}$ & flux term (fieldline integrated source) for plasma energy  & \\
$F^\mathcal{E}$ & flux term (fieldline integrated source divided by field) for plasma energy  &  $m^{-1} s^{-2} C$ \\
$f^n$ & flux term (fieldline integrated source) for plasma number density & \\
$F^n$ & flux term (fieldline integrated source divided by field) for plasma number density  &  $m^{-3} C$ \\
$f^u$ & flux term (fieldline integrated source) for plasma momentum  & \\
$F^u$ & flux term (fieldline integrated source divided by field) for plasma momentum  &  $m^{-2} s^{-1} C$ \\
$f(x,{\bf v},t)$ & generic distribution function & $m^{-6} s^3$ \\
$f_{n,Kn}({\bf v})$ & Knudsen distribution function & $m^{-4} s^4$ \\
$\Gamma(x)$ & gamma function of continuous variable $x$  & \\
$g$ & factor in (twice) the heat flux  & \\
$g(h_j)$ & activation function (of input $h_j$) of a neuron in a neural network  & \\
$G$ & Green's function & \\
$H_\alpha$ & Hamiltonian for species~$\alpha$ & \\
$\hat{\boldsymbol{u}}^e$ &  (K+S) Vector of expansion coefficients & \\
$\hat{v}_g$ &  (K+S) Global list of coefficients & \\
$\hat{v}_g$ &  (K+S) List of all elemental coefficients ($=\underline{v^e}$) & \\
$h$ & mesh or inter-node spacing & $m$ \\
$h_j$ & real-number input to a neuron in a neural network  & \\
$h_p(\xi)$ &  (FE Basis) One-dimensional Lagrange polynomial of order $p$ & \\
$i$ & as suffix denotes ions & \\
$i$ & as suffix denotes regular excited state & \\
$i$ & as suffix generic label & \\
$I$ & as suffix labels Monte-Carlo interactions & \\
$I_\phi$ & $\phi-$ or toroidal component of plasma current & $A$ \\
$I_H$ & Hydrogen reionisation potential as defined in ref~\cite{Ha13Benc} & $eV$ \\
$i,j,k$ &  (K+S) General summation indices & \\
$^I\mathcal{F}_{i\sigma}$ & coefficient of ionisation for the transition from metastable state~$\sigma$ to regular excited state~$i$ & \\
$\in$ &  (Sets) Is a member of; belongs to & \\
$I(\psi)=B_T/R$ & function giving the toroidal field as a function of~$\psi$  & $T m^{-1}$ \\
$I^Z$ & power per atom released in dielectronic recombination  & $W$ \\
$j$ & as suffix is generic label & \\
$j_{ext}(R,Z)$ & electric current density induced in plasma by external coils  & $A m^{-2}$ \\
$j_\phi$ & $\phi-$ or toroidal component of plasma current density  & $A m^{-2}$ \\
$j_{\|}$ &  component of plasma current density parallel to fieldline & $A m^{-2}$ \\
${\bf j}_{sh}$ & sheath plasma current density  & $A m^{-2}$ \\
$k$ & as suffix is generic label & \\
$k$ & chosen to scale so that $kT_0$, $kT_d$ is an energy & $?$ \\
$\kappa_\alpha$ & thermal diffusivity of species~$\alpha$ & $m^2 s^{-1}$ \\
$\kappa_{e\|}$ & parallel thermal diffusivity of electrons & $m^2 s^{-1}$ \\
$\kappa_{e\perp}$ & perpendicular thermal diffusivity of electrons & $m^2 s^{-1}$ \\
$\kappa_{i\|}$ & parallel thermal diffusivity of ions & $m^2 s^{-1}$ \\
$\kappa_{i\perp}$ & perpendicular thermal diffusivity of ions & $m^2 s^{-1}$ \\
$\kappa=k_c/\rho_m c_p$ & thermal diffusivity tensor of solid & $m^2 s^{-1}$ \\
$k_B$ & Boltzmann's constant  & $J K^{-1}$ \\
$k_c$ & thermal conductivity tensor  & $J m^{-1} s^{-1} K^{-1}$ \\
$K_{cx}\left( n_i, T_i \right)$ & reaction rate of charge exchange reactions  & $m^3 s^{-1}$ \\
$K_i$ & ionization reaction rate  & $m^3 s^{-1}$ \\
$K_{MA}$ & chosen as $k_B/m_i$ or $|e|/m_i$ so that $\sqrt{K_MT_d}$ is an ion speed & $?$ \\
$K_M$ & chosen as $k_B/m_u$ or $|e|/m_u$ so that $\sqrt{K_MT_d/A}$ is an ion speed & $?$ \\
$K_r$ & recombination reaction rate  & $m^3 s^{-1}$ \\
$kT_0$ & $T_0$ in energy units  & $J$ \\
$kT_d$ & $T_d$ in energy units  & $J$ \\
$K_v(x)$ & modified Bessel function of the second kind, order $v$  & \\
$k_w$ & wavenumber vector & $m^{-1}$ \\
$\lambda$ & arbitrary quantity  & $?$ \\
$\lambda$ & Coulomb logarithm & \\
$\lambda$ &  (K+S) Helmholtz equation constant & \\
$\Lambda$ & Coulomb logarithm & \\
$\lambda_q$ & $e$-folding length of midplane profile of power loss when an exponential is fitted & $m$ \\
$\Lambda_b$ & sheath potential drop normalized to~$T_e$  & $eV$ \\
$\lambda_D$ & (Glossary) Debye lengthscale above which local electrostatic fluctuations due to presence of discrete charged particles are negligible  & $m$ \\
$\lambda_{mfp,\alpha}$ & (Glossary) Mean free path of particle species~$\alpha$ & $m$  \\
$\langle \sigma v \rangle_{CX}$ & reaction rate for charge exchange  & $m^3 s^{-1}$ \\
$\langle \sigma v \rangle_{ION}$ & reaction rate for ionisation  & $m^3 s^{-1}$ \\
$\langle \sigma v \rangle_{REC}$ & reaction rate for recombination  & $m^3 s^{-1}$ \\
$\langle\sigma v\rangle$ & generic reaction rate  & $m^3 s^{-1}$ \\
$L_0$ & typical lengthscale  & $m$ \\
$L_i^{N_m}(\boldsymbol{\xi})$ &  (FE Basis) Two-dimensional Lagrange polynomial through $N_m$ nodes ${\mathbf \xi}_i$ & \\
$L_s$ & typical lengthscale along fieldline  & $m$ \\
$L_{\|}$ & connection length of typical fieldline  & $m$ \\
$m$ & species particle mass & $kg$ \\
$M_0$ & Mach number at $s=0$ boundary & \\
$M_1$ & Mach number at $s=1$ boundary & \\
$m_\alpha$ & mass of species~$\alpha$ & $kg$ \\
$\mathbb{E}$ & expectation  & \\
$\mathbb{E}_{k\neq i, l\neq j}$ & expectation computed by integrating over all the $x_k$ except for~$x_i$ and  $x_j$  & \\
$\mathbb{E}_{x_{k\neq i}}$ & expectation computed by integrating over all the $x_k$ except for~$x_i$  & \\
$\mathbb{L}(u)$ &  (K+S) Linear operator in $u$ & \\
$\mathbb{P}$ &  (K+S) Projection operator & \\
$\mathbb{P}^{\delta}$ &  (K+S) Discrete projection operator & \\
${\mathbf v}$ &  (K+S) Velocity $[u, v, w]^T$ & \\
$\mathcal{E}_\alpha$ & energy of species~$\alpha$ & $J m^{-3}$ \\
$\mathcal{E}_e$ & energy of the electrons & $J m^{-3}$ \\
$\mathcal{E}_i$ & energy of the ion species & $J m^{-3}$ \\
$\mathcal{E}_R$ & total plasma radiation & $W m^{-3}$ \\
$\mathcal{F}$ & generic coefficient of excitation, ionisation or recombination  & $m^3 s^{-1}$ \\
$\mathcal{F}_\alpha$ & functional of moments of species~$\alpha$  & $m^{-6} s^3$ \\
$\mathcal{I}$ &  (K+S) Interpolation operator & \\
$\mathcal{I}^{\delta}$ &  (K+S) Discrete interpolation operator & \\
$\mathcal{K}_{\|}$ & parallel thermal conductivity of plasma & $m^{-1}s^{-1}$ \\
$\mathcal{K}$ & thermal conductivity of plasma & $m^{-1}s^{-1}$ \\
$\mathcal{K}_{\perp}$ & thermal conductivity of plasma perpendicular to field and flux surface & $m^{-1}s^{-1}$ \\
$\mathcal{K}_{\wedge}$ & thermal conductivity of plasma perpendicular to field in flux surface & $m^{-1}s^{-1}$ \\
$\mathcal{L}_7$ & 7-D Lie derivative (space, velocity-space and time make up the $3+3+1=7$ dimensions)  & $s^{-1}$ \\
$\mathcal{P}_{P}(\Omega)$ &  (K+S) Polynomial space of order $P$ over $\Omega$ & \\
$\mathcal{Q}$ & coefficient in radiation equation  & $m^3 s^{-1}$ \\
$\mathcal{Q}_{\sigma \rightarrow \rho}^{Z\rightarrow Z}$ & parent-metastable cross-coupling coefficient  & $m^3 s^{-1}$ \\
$\mathcal{S}$ & coefficient in radiation equation  & $m^3 s^{-1}$ \\
$\mathcal{S}^{Z_m\rightarrow Z}$ & ionisation coefficient  & $m^3 s^{-1}$ \\
$\mathcal{S}^{Z\rightarrow Z_p}$ & ionisation coefficient  & $m^3 s^{-1}$ \\
$\mathcal{T}$ & generic tensor & $?$ \\
$\mathcal{V}$ &  (K+S) Space of test functions & \\
$\mathcal{V}^{\delta}$ &  (K+S) Finite-dimensional space of test functions & \\
$\mathcal{X}$ & coefficient in radiation equation  & $m^3 s^{-1}$ \\
$\mathcal{X}_{\sigma \rightarrow \rho}^{Z\rightarrow Z}$ & generalised collisional-radiative (GCR) excitation coefficient  & $m^3 s^{-1}$ \\
$\mathfrak{R}$ & Real numbers & \\
$\mathrm{Var}(f)$ & variance of the distribution of $f$ computed by integrating over all variables~$x_i$  & \\
$\mathrm{Var}[Q]$ & variance in random variable $Q$  & \\
$m_e$ & mass of electron & $kg$ \\
$m_i$ & mass of ion species particle $m_i= A m_u$ & $kg$ \\
$m_n$ & neutral species particle mass & $kg$ \\
$m_p$ & mass of proton & $kg$ \\
$m_u$ & atomic mass unit & $1.6605 \times 10^{-27}\,kg$ \\
$M_s$ & Mach number, allowed to take either sign & \\
$M_S$ & number of energy states of an atom & \\
$\mu, \nu$ &  (K+S) Dynamic, kinematic viscosities & \\
$\mu_{cx}=\omega_c/\nu_{cx}$ & measures strength of magnetization with respect to charge exchange reaction & \\
$\mu_m$ & reduced mass of two particles & $kg$ \\
$M_Z$ & number of metastable states for species~$\alpha$ (which includes the ground state) & \\
$n$ & number density & $m^{-3}$ \\
$n_{ref}$ & reference number density of the plasma ions & $m^{-3}$ \\
$N_{ref} $ & normalising or reference number density & $10^{18}$  \\
$N$ & number density, may be scaled by $N_{ref}=10^{18}$ & $m^{-3}$ \\
$n_0$ & initial number density & $m^{-3}$ \\
$\nabla \cdot$ &  (K+S) Divergence & \\
$\nabla \times$ &  (K+S) Curl & \\
$\nabla^2$ &  (K+S) Laplacian & \\
$N_{b}$ &  (K+S) Number of global boundary degrees of freedom & \\
$n_B=N/B$ & number density divided by field strength & $m^{-3}T^{-1}$ \\
$N_D$ &  Number of degrees of freedom per dimension, $D=1,2,\ldots6$ & \\
$N_{dof}$ &  (K+S) Number of global degrees of freedom & \\
$n_{con}$ & blob contrast factor  &  \\
$n_e$ & number density of the electrons & $m^{-3}$ \\
$N_{el}$ &  (K+S) Number of finite elements & \\
$N_{eof}$ &  (K+S) Total number of elemental degrees of freedom $N_{eof} \simeq N_{el}N_m$ & \\
$n_i$ & number density of the plasma ions & $m^{-3}$ \\
$n_j({\bf x},t)$ & member of the set of deterministic coefficients of the ``random trial basis" & \\
$N_{m}$ &  (K+S) Number of elemental degrees of freedom & \\
$n_n$ & neutral density & $m^{-3}$ \\
$\notin$ &  (Sets) Is not a member of; does not belong to & \\
$\not\subset$ &  (Sets) Is not a subset of & \\
$n_p$ & number density of the plasma ions & $m^{-3}$ \\
$N_{Q}$ &  (K+S) Total number of quadrature points $N_Q = Q_1 Q_2 Q_3$ & \\
$n_s$ & number density of isotope~$s$ & $m^{-3}$ \\
$N_s$ & number density of isotope~$s$ & $m^{-3}$ \\
$N_T$ & number of samples in temperature used to define typically a crossection in the ADAS database~\cite{adaswebsite,openadaswebsite} & \\
$\nu$ & plasma kinematic viscosity & $m^2 s^{-1}$ \\
$\nu_\alpha$ & kinematic viscosity of species~$\alpha$ & $m^2 s^{-1}$ \\
$\nu_{cx}=K_{cx} n_n$ & charge exchange `frequency'  & $s^{-1}$ \\
$\nu_{e0}$ & electron kinematic viscosity caused by neutrals & $m^2 s^{-1}$ \\
$\nu_{e\|}$ & parallel kinematic viscosity of electrons & $m^2 s^{-1}$ \\
$\nu_{e\perp}$ & perpendicular kinematic viscosity of electrons & $m^2 s^{-1}$ \\
$\nu_{i\|}$ & parallel kinematic viscosity of ions & $m^2 s^{-1}$ \\
$\nu_i$ & ion kinematic viscosity & $m^2 s^{-1}$ \\
$\nu_{i0}$ & ion kinematic viscosity caused by neutrals & $m^2 s^{-1}$ \\
$\nu_{i\perp}$ & perpendicular kinematic viscosity of ions & $m^2 s^{-1}$ \\
$\nu^{*}_\alpha$ & (Glossary) Normalised collision frequency for species~$\alpha$ & \\
$\nu^{*}_c = \frac{q_e^4}{3 m_p^2\epsilon_0^2} L_0 n_0 /C_0^4$ & Collisionality parameter & \\
$\nu_\alpha$ & (Glossary) Collision frequency for species~$\alpha$ & $s^{-1}$ \\
$\nu_{\alpha{\sf n}}$ & Collision frequency for species~$\alpha$ with neutrals & $s^{-1}$ \\
$\nu_{\alpha\beta}$ & Collision frequency for species~$\alpha$ with species $\beta$ & $s^{-1}$ \\
$n^Z$ & number density for charge state~$Z$  & $m^{-3}$ \\
$N_P$ & number of particles in a calculation & \\
$N_{P\alpha} $ & number of particles of species $\alpha$ in a calculation & \\
$N_Z$ & number of charge states for an ion species & \\
$n^Z_i$ & number density for charge state~$Z$, excited state~$i$  & $m^{-3}$ \\
$n^Z_\sigma$ & number density for charge state~$Z$, metastable state~$\sigma$  & $m^{-3}$ \\
$\omega_{ce}= |e|B/m_e$ & electron cyclotron angular frequency & $radians s^{-1}$ \\
$\omega_{ci}= Z_i e B/m_i$ & ion cyclotron angular frequency & $radians s^{-1}$ \\
$\omega_{pe}= \sqrt{\frac{nq_e^2}{\epsilon_0 m_e}}$ & plasma angular frequency for electrons & $radians s^{-1}$ \\
$\omega_{pi}= Z_i\sqrt{\frac{nq_e^2}{\epsilon_0 m_i}}$ & plasma angular frequency for ions & $radians s^{-1}$ \\
$\Omega$ &  (K+S) Solution domain & \\
$\Omega^e$ &  (K+S) Elemental region & \\
$\mathrm{p}(A|B)$ & conditional probability of event $A$ given event $B$ is known or assumed to have occurred  & \\
$p_\alpha$ & pressure of species~$\alpha$  & $N m^{-2}$ \\
$\parallel  Q \parallel_E $ &the `energy' norm  & \\
$(\partial f/\partial t)_C$ & source in Boltzmann due to inter-particle interactions  & $m^{-6} s^2$ \\
$\partial \Omega_e$ &  (K+S) Boundary of $\Omega^e$ & \\
$\partial \Omega$ &  (K+S) Boundary of $\Omega$ & \\
$\partial \Omega_{\mathcal D}$ &  (K+S) Domain boundary with Dirichlet conditions & \\
$\partial \Omega_{\mathcal N}$ &  (K+S) Domain boundary with Neumann conditions & \\
$P_C$ & number of modes in basis for polynomial chaos & \\
$p_e$ & pressure of the electrons  & $N m^{-2}$ \\
$\phi$ & angle in toroidal direction & radians $^c$ \\
$\Phi$ & electr(ostat)ic potential  & $V$ \\
$\phi_{pq}, \phi_{pqr}$ &  (FE Basis) Expansion basis & \\
$\phi_{e,\xi}$ & (FE Basis) expansion basis as a function of global position~${\bf x}$ & \\
$p$ & (K+S) pressure  & $N m^{-2}$ \\
$p = \sum_\alpha n_\alpha kT_\alpha$ & plasma pressure  & $N m^{-2}$ \\
$p$ & as suffix labels (super-)particles & \\
$p_i$ & pressure of the ion species  & $N m^{-2}$ \\
$P_i$ &  (FE Basis) Polynomial order in the $i$th direction & \\
$p(\psi)$ & function giving the pressure as a function of~$\psi$ of the magnetic flux  & $N m^{-2}$ \\
$p,q,r$ &  (K+S) General summation indices & \\
$Pr$ & Prandtl number & \\
$Pr_M$ & magnetic Prandtl number & \\
$\psi$ & poloidal magnetic flux  & $T m^2$ \\
$\psi^a_p, \psi^b_{pq}, \psi^c_{pqr}$ &  (FE Basis) Modified principal functions & \\
$\Psi_i$ & $i^{th}$ member of a set of basis functions, typically multi-dimensional Hermite polynomials & \\
$P(T)$ & emitted power integrated over all wavelengths  & $W m^3$ \\
$\mathrm{p}(x)$ & probability distributions    & \\
$P(x)$ & Cumulant probability distribution  & \\
$P^Z$ & radiated power per atom of $n^Z$  & $W$ \\
$Q_\|$ & combined energy flux at a boundary  & $J m^{-2} s^{-1}$ \\
$q_\alpha$ & charge on a particle of species~$\alpha$ & $C$ \\
$q_e$ & charge on an electron, negative by convention & $C$ \\
$Q(f_\alpha, f_\beta)$ & Boltzmann collision operator  & $m^{-6} s^2$ \\
$Q_H$ & cooling rate due to excitation as defined in ref~\cite{Ha13Benc}  & $K m^{-3} s^{-1}$ \\
$q_i$ & charge on an ion  & $C$ \\
$q_{\|e}$ & electron energy flux along fieldline & $J m^{-2} s^{-1}$ \\
$q_{\|i}$ & ion energy flux along fieldline & $J m^{-2} s^{-1}$ \\
${\bf q}_e$ & electron energy flux  & $J m^{-2} s^{-1}$ \\
${\bf q}_i$ & ion energy flux  & $J m^{-2} s^{-1}$ \\
$Q_i$ &  (FE Basis) Quadrature order in the $i$th direction & \\
$Q_{ie}$ & collisional energy equipartition term  & $kg m^{-1} s^{-3}$ \\
$r$ & order of higher order term  & \\
$r_0$ & radius used in initial condition, such as blob size & $m$ \\
$R$ & cylindrical coordinate  & $m$ \\
$R_0$ & major radius of torus & $m$ \\
$R_p$ & recycling coefficient for particles & \\
$R_E$ & recycling coefficient for particle energy & \\
$\rho$ & as suffix is label of metastable state & \\
$\rho$ &  (K+S) Density & \\
$\rho_c=\sum_\alpha Z_\alpha |e| n_\alpha$ & charge density of the medium  & $C m^{-3}$ \\
$\rho_m=\sum_\alpha A_\alpha m_u n_\alpha$ & mass density of the medium  & $kg m^{-3}$ \\
$\rho_{t\alpha}$  & (Glossary) Gyroradius or Larmor radius of orbit of charged particle of species~$\alpha$ about magnetic field direction  & $m$ \\
$^R\mathcal{F}_{i\sigma}$ & coefficient of recombination for the transition from metastable state~$\sigma$ to regular excited state~$i$  & \\
$s_{\|}$ & arclength along fieldline  & $m$ \\
$s$ & as suffix, isotope label ($\alpha$ preferred for species) & \\
$s$ & parameterises distance along the fieldline $0\leq s \leq 1$ & \\
$S_\alpha$ & source term in Boltzmann equation for species~$\alpha$  & $m^{-6} s^2$ \\
$S_C$ & total source term in Boltzmann equation  & $m^{-6} s^2$ \\
$S_\mathrm{ana}({\bf x},t)$ & explicit/analytic source term in fluid equation(s)  & $m^{-3} s^{-1}$ ? \\
$S^n_\mathrm{ana}({\bf x},t)$ & numerically convenient source term in fluid equation(s)  & $m^{-3} s^{-1}$ ? \\
$S_{exp}({\bf x}, {\bf v},t)$ & explicit source term in Boltzmann equation  & $m^{-6} s^2$ \\
${\sf n}$ & neutral density  & \\
${\sf T}$ & neutral temperature  & \\
${\sf u}$ & neutral velocity  & \\
$s_i$ & arclength parameter for boundary ($i=1$ inner, $i=2$ outer) & \\
$s^\mathcal{E}$ & source term in plasma energy equation  &  \\
$s^{\mathcal{E}}_{e}$ & energy density source term for electrons  &  \\
$s^{\mathcal{E}}_{i}$ & energy density source term for ions  &  \\
$s^\mathcal{E}_n$ & source term in neutral energy equation  &  \\
$s^\mathcal{E}_{\perp e}$ & energy cross-field source term for electrons  &  \\
$s^\mathcal{E}_{\perp i}$ & energy cross-field source term for ions   &  \\
$s^\mathcal{E}_{\perp n}$ & energy cross-field source term for neutrals  &  \\
$s^n$ & source term in plasma density equation  &  \\
$s^n_n$ & source term in neutral density equation  &  \\
$s^n_{e}$ & number density source term for electrons  &  \\
$s^n_{i}$ & number density source term for ions  &  \\
$s^u$ & source term in plasma momentum equation  &  \\
$s^u_n$ & source term in neutral momentum equation   &  \\
$s^u_{\perp n}$ & momentum cross-field source term for neutrals  &  \\
$S_i$ & Sobol sensitivity index, gives a normalised measure of the sensitivity of the distribution of $f$ to the parameter~$x_i$  & \\
$\sigma$ & as suffix labels metastable state & \\
$\sigma$ & reaction cross-section  & $m^2$ \\
$\sigma_C$ & reaction rate for charge exchange  & \\
$\sigma_E$ & cooling rate due to excitation  & \\
$\sigma_E$ & electrical conductivity  & $\Omega^{-1} m^{-1}$ \\
$\sigma_I$ & reaction rate for ionisation  & \\
$\sigma_s^{i|0}$ & collision cross-section for ions with neutrals  & $m^2$ \\
$\sigma_s^{e|0}$ & collision cross-section for electrons with neutrals  & $m^2$ \\
$S_{ij}$ & Sobol sensitivity index, gives a normalised measure of the sensitivity of the distribution of $f$ to the parameters~$x_i$ and $x_j$  & \\
$S^\mathcal{E}$ & source term in plasma energy equation  & $kg m^{-1} s^{-3}$ \\
$S^{\mathcal{E}}_{e}$ & energy density source term for electrons  & $kg m^{-1} s^{-3}$ \\
$S^{\mathcal{E}}_{i}$ & energy density source term for ions  & $kg m^{-1} s^{-3}$ \\
$S^\mathcal{E}_n$ & source term in neutral energy equation  & $kg m^{-1} s^{-3}$ \\
$S^\mathcal{E}_{\perp e}$ & energy cross-field source term for electrons  & $kg m^{-1} s^{-3}$ \\
$S^\mathcal{E}_{\perp i}$ & energy cross-field source term for ions   & $kg m^{-1} s^{-3}$ \\
$S^\mathcal{E}_{\perp n}$ & energy cross-field source term for neutrals  & $kg m^{-1} s^{-3}$ \\
$S^n$ & source term in plasma density equation  & $m^{-3} s^{-1}$ \\
$S^n_{e}$ & number density source term for electrons  & $m^{-3} s^{-1}$ \\
$S^n_{i}$ & number density source term for ions  & $m^{-3} s^{-1}$ \\
$S^n_n$ & source term in neutral density equation  & $m^{-3} s^{-1}$ \\
$S^n_{\perp n}$ & number density cross-field source term for neutrals  & $m^{-3} s^{-1}$ \\
$S^n_{\perp}$ & number density cross-field source term for plasma  & $m^{-3} s^{-1}$ \\
$S_{\perp n}$ & generic cross-field source term for neutrals  & $m^{-3} s^{-1}$ \\
$S^u$ & source term in plasma momentum equation  & $kg m^{-2} s^{-2}$ \\
$\subset$ &  (Sets) Is a subset of & \\
$S^u_n$ & source term in neutral momentum equation   & $kg m^{-2} s^{-2}$ \\
$S^u_{\perp n}$ & momentum cross-field source term for neutrals  & $kg m^{-2} s^{-2}$ \\
$S^Z_\rho$ & particle source for ion of metastable state~$\sigma$ (species~$\alpha$) with charge state~$Z$  & $m^{-3} s^{-1}$ \\
$S^Z_\alpha$ & particle source for ion of species~$\alpha$ with charge state~$Z$  & $m^{-3} s^{-1}$ \\
$t$ & time usually in seconds  & $s$ \\
$t'$ & offset time usually in seconds  & $s$ \\
$T$ & plasma temperature  & $eV$ \\
$t_0$ & characteristic evolutionary timescale usually in seconds  & $s$ \\
$t_s$ & characteristic timescale usually in seconds  & $s$ \\
$t_H$ & Numerical hand-off time interval usually in seconds  & $s$ \\
$t_R$ & Numerical ramp-up time interval usually in seconds  & $s$ \\
$T_0$ & initial temperature (prefixed by $k$ implies energy in SI)  & $eV$ \\
$T_{Kn}$ & reference temperature of Knudsen distribution (prefixed by $k$ implies energy in SI)  & $eV$ \\
$T_{ref}$ & reference temperature (prefixed by $k$ implies energy in SI)  & $eV$ \\
$T_s$ & characteristic temperature ($T_s=(L_s/t_s)^2/K_M$)  & $eV$ \\
$T_\alpha$ & temperature of species~$\alpha$  & $eV$ \\
$\tau$ & optical depth  & $m$ \\
$\tau_\alpha$ & collision or relaxation time of species~$\alpha$   & $s$ \\
$\tau_e$ & electron collision or relaxation time   & $s$ \\
$\tau_i$ & ion species collision or relaxation time   & $s$ \\
$\tau_{e{\sf n}} $ & electron-neutral collision time   & $s$ \\
$\tau_{i{\sf n}} $ & ion species-neutral collision time   & $s$ \\
$\tau_{ce}=1/f_{ce}$ & electron cyclotron timescale & $s$ \\
$\tau_{ci}=1/f_{ci}$ & ion cyclotron timescale & $s$ \\
$\tau_{pe}=1/f_{pe}$ & plasma timescale for electrons & $s$ \\
$\tau_{pi}=1/f_{pi}$ & plasma timescale for ions & $s$ \\
$\tau_{\mathcal{E}_e}$ & loss time of energy density for electrons   & $s$ \\
$\tau_{\mathcal{E}_i}$ & loss time of energy density for ions   & $s$ \\
$\tau_{n_e}$ & loss time of number density for electrons   & $s$ \\
$\tau_{n_i}$ & loss time of number density for ions   & $s$ \\
$T_d=T_i+T_e$ & combined temperature of the electrons and ions  & $eV$ \\
$T_e$ & electron temperature (prefixed by $k$ implies energy in SI)  & $eV$ \\
$T_H$ & the Hydrogen reionisation potential  & \\
$\theta$ & angular coordinate & radians $^c$ \\
$\theta$ & random parameter~$0\leq\theta\leq 1$ & \\
$T_i$ & ion temperature  & $eV$ \\
$\tilde{a}$ & scaled matrix coefficient & \\
$\tilde{b}=B/B_0$ & dimensionless magnetic field & \\
$\tilde{\psi}^a_p, \tilde{\psi}^b_{pq}, \tilde{\psi}^c_{pqr}$ &  (FE Basis) Orthogonal principal functions & \\
$u$ & generic first velocity component  & $m s^{-1}$ \\
$U$ & velocity component (flow) along fieldline  & $m s^{-1}$ \\
$U_\alpha$ & velocity component (flow) along fieldline of species $\alpha$ & $m s^{-1}$ \\
$U_d =L_s/t_0$ & speed measuring the importance of the transient term  & $m s^{-1}$ \\
$U_s =L_s/t_s$ & characteristic speed  & $m s^{-1}$ \\
$U_A$  & Alfv\'{e}n speed  & $m s^{-1}$ \\
$\underline{\boldsymbol{f}^e}$ &  (K+S) Concatenation of elemental vector $\boldsymbol{f}^e$ & \\
$\underline{\boldsymbol{W}^e}$ &  (K+S) Block-diagonal extension of matrix $\boldsymbol{W}^e$ & \\
$u_R=1/R$ & Radial component of Grad-Shafranov `flow' & \\
$v$ & generic second velocity component  & $m s^{-1}$ \\
$v_{\|}$ & fluid velocity component along fieldline  &  $m s^{-1}$ \\
$V^e$ & spatial volume occupied by finite element~$e$  & $m^3$ \\
$V_i$ & variance of the distribution of $f$ as the parameter~$x_i$ varies  & \\
$V_{ij}$ & variance of the distribution of $f$ as the parameters~$x_i$ and $x_j$ vary  & \\
$w$ & generic third velocity component  & $m s^{-1}$ \\
$w_{jk}$ & weight in neural network indexed by neuron $j$ and input $k$  & \\
$w_p$ & weight of particle $p$ & \\
$w_{\alpha,ref}$ & normalising or reference weight of particle of species~$\alpha$ & \\
$w_{ref} $ & normalising or reference number for superparticles & $10^{10}$  \\
$W$ & weighting function for particle-in-cell & \\
$x$ & Cartesian coordinate  & $m$ \\
$x_0$ & coordinate value used in specifying initial condition, eg.\ blob position & $m$ \\
$x_1, x_2, x_3, {\mathbf x}$ &  (FE Basis) Global Cartesian coordinates & \\
$x_\alpha$ & collisionality factor of species~$\alpha$ & \\
$x_e = \omega_{ce}\tau_e$ & collisionality factor of electrons & \\
$x_i = \omega_{ci}\tau_i$ & collisionality factor of ions & \\
$x_i$ & generic parameter or variable  & \\ 
$\xi_1, \xi_2, \xi_3, \boldsymbol{\xi}$ &  (FE Basis) Local Cartesian coordinates & \\
$\xi_i$ & random number within the unit interval~$[0,1]$  & \\
$^X\mathcal{F}_{i\sigma}$ & coefficient of excitation for the transition from metastable state~$\sigma$ to regular excited state~$i$  & \\
$y$ & Cartesian coordinate  & $m$ \\
$y_0$ & coordinate value used in specifying initial condition, eg.\ blob position & $m$ \\
$z$ & Cartesian coordinate  & $m$ \\
$z_0$ & coordinate value used in specifying initial condition & $m$ \\
$Z$ & Cartesian coordinate  & $m$ \\
$Z$ & charge state of the ion & \\
$Z$ & cylindrical coordinate  & $m$ \\
$Z_0(\alpha)$ & number of charge states of species~$\alpha$ included in the model & \\
$Z_a$ & Gaussian random process, index $a$  & \\
$\zeta$ & magnetic Prandtl number as defined in Cambridge & \\
$\zeta=-\phi$ & toroidal angle coordinate & radians $^c$ \\
$Z_{eff}$ & effective charge state of plasma ions & \\
$Z_i$ & charge state of ion & \\
$Z_\alpha$ & charge state of ion species $\alpha$ & \\
$Z_m=Z-1$ & where $Z$ is ion charge state & \\
$Z_p=Z+1$ & where $Z$ is ion charge state & \\
$Z_{sum}=\sum_\alpha Z_0(\alpha)$ & where $Z_0$ is number of charge states of species~$\alpha$ & \\
\hline
\end{longtable}

