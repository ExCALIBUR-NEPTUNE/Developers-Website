\subsection{Introduction}
\label{sec:introduction}

This section is Project \nep's response to Tokamak Science and MAST Upgrade Division's
Requirements Specification document.
This specification was received by Wayne Arter  on 5/11/20 via Rob Akers
and its authorship was confirmed to be Fulvio Militello and James Harrison at the \nep \  Project Board.
It is intended that this document helps to frame expectations of the
capabilities of \nep\ code,
as well as highlighting challenges and areas of potential collaboration.
Since  the main focus herein is on aspects of the physical model, the
next \Sec{TS_sw_response} goes into more detail regarding the software engineering
needed to deliver the code successfully.

The physics requirements are summarised as:
\begin{quote}
``The new UKAEA Exhaust code needs to be able to capture parallel and
perpendicular transport of charged and neutral particles in 3D, full geometry
and in a time dependent way. Turbulence should be self-consistently modelled,
as well as energy transfer physics between charged particles, neutrals and
photons (radiation). While perturbations need to be 3D, a minimal requirement
for the code is that it can simulate realistic axisymmetric equilibrium
configurations with complex topologies and wall designs.
The aim of the code should be to:
\begin{enumerate}
	\item Efficiently and reliably model exhaust in next generation
		experiments, like eg.\  MAST-U, JT60-SA, and especially ITER
		(the latter is a stringent requirement for the code).
	\item Allow predictive exhaust capability for future reactor relevant
		machines like STEP or DEMO.''
\end{enumerate}
\end{quote}
More detailed specifications are given in the following sections
either as block quotations or as bold paragraph headers.

\subsubsection{\nep\ Science Plan}

The Science Plan for \nep\ \cite{sciplan} is available online.
The stated goal of the project is to:
{\green 
\begin{quote}
	``develop new algorithms, software and related e-Infrastructure that will result in the efficient use of current Petascale and future Exascale supercomputing hardware in order to
\begin{enumerate}
	\item draw insights from ITER ``Big Data''
	\item to guide and optimise the design of the UK demonstration nuclear
		fusion power plant STEP and related fusion technology
\end{enumerate}
in the approach to the Exascale. The aims of the work are to deliver expertise in,
and tools for, ``in-silico'' reactor interpretation and design.''
\end{quote}}

The Science Plan also describes the software development and theory development
that is being and will be undertaken under \nep.

\paragraph{Software development.}
The aim of \nep\ software development is to provide a flexible framework for
implementing different physical models in an Exascale-targeted manner.
In particular, the project does \emph{not} envisage a ``\nep\ system of equations'' so
much as \nep\ providing the ability to solve a class of relevant equations,
with models described relatively simply using a Domain-Specific Language (DSL).
This flexibility enables the hierarchy of models of varying fidelity and
computational costs.
It also allows engagement from different classes of user with different levels
of physics and software expertise.

\paragraph{Theory development.}
The above framework approach notwithstanding,
\nep\ is also supporting theory development in two of its four work packages:
FM-WP2 Plasma Multiphysics Model and FM-WP3 Neutral Gas and Impurity Model.
These will develop two close coupled models, with FM-WP2 seeking to include
kinetic effects in existing and new edge plasma models, and FM-WP3 developing
particle-based models for describing the region outside and just inside the
plasma (neutral atoms/molecules and partially ionised impurities).

Another work package, FM-WP1 Numerical Representation, is addressing related
numerical issues, such as
the accurate modelling of highly anisotropic dynamics,
the accurate representation of first wall geometry,
and the numerical preservation of conservation laws from the underlying models.

As such the \nep\ plan is aligned with all the points in the summary
requirements quoted above, though there are minor issues regarding
the details as discussed in the following sections.
It should be noted that the Science Plan outlines a five-year programme that
explicitly requires user involvement for fuller development of surrogate
models (such as turbulent friction) and to specify the detailed physics of
ionisation and excitation reactions.
It follows that development of many of the physics capabilities listed below will benefit
greatly from a strong collaboration between
Tokamak Science and Advanced Computing.
