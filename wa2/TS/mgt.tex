%\section{Technical Specification}\label{sec:TS_MGT}

\subsection{Code licence and availability} \label{sec:lic}

Code should be made available to collaborators at the earliest opportunity, to maintain
close alignment between groups. 

\begin{itemize}
\item To minimise friction and unnecessary legal restrictions on combining
code components, a common, MIT licence has been adopted. This may be regarded 
as equivalent to the BSD3 licence in the \nep \ Charter, see~\Sec{charter}.

\item
Code development should be carried out in public
repositories. The benefits of minimising delay to code use, feedback
and peer review, outweigh any potential for embarrassment or code misuse.
\end{itemize}

\subsection{Code style} \label{sec:style}

There are many different code styles, each of which have their
proponents, and can be debated at length.  While everyone has their
own favourite style, it seems likely that the choice of style makes little
difference to objective quality or productivity.  Anecdotal experience
and experience from the gaming world in developing large, complicated packages
(eg.\ Gregory~\cite[\S\,3]{gregory}),
indicate however that it is very important that there is a well-defined
code style and that developers stick to it, since a mixture of
styles in a code base adds unnecessary mental load and overhead.
The ultimate choice is down to the project Lead, under advice from
major code contributors, leading websites and textbooks, eg.\ in the
case of the C++ language, the C++ core guidelines~\cite{cppguidelines}
and more compactly Stroustrop's ``Tour of of C++"~\cite{stroustroptour}.
 
The following style choices have been made and efforts will be made to enforce them
in \nep \ repositories.
\begin{enumerate}
\item Formatting of C++ and Python will follow the style used by Nektar++
\begin{itemize}
\item Developer tools: Code formatting tools should be used to
automatically format code.  For C++ \T{clang-tidy}~\cite{clang-tidywebsite}
should  be used while, and for Python \T{Black}~\cite{blackwebsite}
is recommended. Similar tools should be chosen
for any other languages adopted by the project.

\item Enforcement: Tests run on pull requests and code pushes to the
shared repository should include code formatting: The automated
formatter is applied to the code, and if the output is different
from the input then the code is incorrectly formatted, and the test
fails.

\item Documentation: \LaTeX \ or Markdown should be used
and conventions enforced regarding a maximum line length of $120$~characters,
restriction to ASCII character set, abbreviations, hyphenation,
capitalisation, minimal use of~`z', and use of fonts to denote code names.
The \LaTeX \ {\tt lwarp} package is used to generate this website.

\end{itemize}

\item Naming 

Naming of code components (modules, classes, functions, variables
etc.) is less easy to enforce automatically than formatting, but is
arguably essential for \nep, because of the complexity of the physical
processes to be modelled.
Widely applicable good practices to be adopted are as follows:
\begin{itemize}
\item Consistency: Whatever convention is used, stick to it.
\item Be descriptive: Names should be meaningful, not cryptic, and need
not be very short, although brevity consistent with frequency of usage
is recommended.
\begin{itemize}
\item Variable names, used only on input, might consist of $3-4$ words primarily describing
the variable, using `pot-hole' convention
\item Loop or indexing variables, might consist of single characters, eg.\ \T{i}, \T{j}, \T{k}
\item Mathematical symbols, eg.\ those defined in \Sec{symbol}, should be converted 
from \LaTeX \ into variable names using the conventions of \Sec{conv}.
\item Where mathematical symbols are converted for use in larger expressions, 
the mathematical expression
should be in the documentation within or linked to  the code.
\end{itemize}
\item Generally prefer nouns for variables, and verbs for functions
\end{itemize}
%Some code styles for C++ (eg.\ BOUT++, adapted from LLVM and with features of the 
%the JSF coding style~\cite[\S\,6.6]{pittwhitely}), use different cases
%for different types of things: \T{snake\_case} for variables, \T{camelCase}
%for functions/methods, and \T{PascalCase} for classes and types.
Given a need to mix with Object Fortran, which is not case-sensitive,
the `pot-hole' convention for naming C++ variables, ie.\ separating
name elements by underscores, is recommended.

Occasionally a convention is used where the name includes a part which
indicates the type of the variable. For example, the JSF style for C++~\cite[\S\,6.6]{pittwhitely}
recommends that pointer names begin~`p\_' and that private or protected (`member') variables
should have names beginning with~`m'.
In general naming conventions are probably not
essential, since the type can be read in the code, and modern IDEs will
easily provide this information to the developer.
Nonetheless, there is no objection to employing a convention, and \nep \
recommend the following one for coders who wish to do so.

For C++, based on the recommendations of the book ``Professional C++" (eg.\
\cite[\S\,7]{solterkleper}, the following prefix strings should be employed:
`m\_' for  member (particularly useful for indicating scope),
`p\_' for pointer, `s\_' for static, `k\_' for constant,
`f\_' for flag (Boolean value),
`d\_' for buffers/pointers on an accelerator device,
and aggregations thereof, eg.\ `ms\_variable'.
Use of global variables is deprecated, so the `g\_' prefix should not be used.

For Object Fortran naming conventions, Arter~et~al~\cite{fprog} codifies best practice.
%More useful is a
%means of instantly identifying the scope of the variable, ie.\ non-local (default) or
%local (prefix `z'),  and whether its value is variable (default) or fixed (static~C++,
%prefix `k'). 
It may also be useful to reserve the single letters~`i', `j', `k' etc. for the
names of loop-count variables. Loop counting should start at unity, consistent
with everyday practice, unless there are good reasons for starting at zero
or using offset values. % (which should normally start with a value of unity).

\end{enumerate}

\subsection{Programming languages} \label{sec:lang}

A small set of ``approved" languages is to be used in the project
consistent with the rule of `two' described in \Sec{MGT_intro}.
This rule covers the high performance code itself and
also the input/output, testing scripts and other infrastructure included in
the code repository.

Important factors in the choice made included:
\begin{enumerate}
\item Widespread use. It must be possible for several project members at any one time
to understand the language, and be able to maintain the code.
\item Stability. The code developed will potentially have a long life-span, 
and there are insufficient resources to continually update code to respond to
upstream changes.
\item Previous usage in HPC and scientific computing. There should be an
existing ecosystem of code packages, tutorials, and potential users.
\end{enumerate}
The above considerations implied that the following options were extensively discussed.
\begin{itemize}
\item C++14, Fortran (eg.\ 2008), C and Python all satisfy the above criteria. 
For configuration, CMake, Autotools and Bash also qualify.
\item SYCL (building on C++) and Julia are both less widely adopted so
far, but both appear to be heading towards satisfying the above
criteria and  might be considered.
\end{itemize}

The recommended languages are
\begin{itemize}
\item As higher level DSL : Python and Julia
\item For lower level HPC compatibility/DSL: Kokkos and SYCL
\item General scientific work : the latest versions of C++ and (Object) Fortran, provided
they are compatible with pre-existing packages and reliable compilers are available
(eg.\ as of mid-2021 usage of SYCL implies a need for C++17.)
\item For code compilation and linking etc.\ : CMake
\end{itemize}

Other languages may have technical merits in particular areas, or are
being adopted outside scientific computing but not to a significant
degree within the community.  Use of these languages should be limited
to isolated experiments, rather than core components. If shown to be
useful in these experiments, to a level which is worth the additional
overhead and risk of maintaining it, then the Lead should consider
expanding the list of approved languages, consistent with the `rule of two'.

