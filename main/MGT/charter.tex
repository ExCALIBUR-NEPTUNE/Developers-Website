All members of the \exc \   \nep \   team should be aware that 
to meet the challenges of the \nep \   project, and the \exc \   overarching 
pillars, a distributed team of scientists, software engineers and architecture 
specialists from different UK institutions will be required to form a community 
around the \nep \  project (and will connect across the overarching \exc \   
programme). A high-level objective is to ensure that developed software is of 
the highest quality, implying a rigid requirement around the production of 
high-quality documentation and reproducible verification and validation tests 
for the codebase as it evolves. Since development work may transfer between 
institutions, it is important that common standards for documentation and 
testing be available and easy to deploy. The initial \nep \   exploratory 
\Papp s~may be written in a range of languages including for example Python, 
C++/DPC++, Object Fortran or Julia, however it is envisaged that there will be 
an emerging steer towards a reduced set of languages and technologies to ensure 
interoperability across the \nep \   software stack, ultimately leading to 
coupled simulations covering all the physics necessary to deliver an 
``actionable'' simulation for the plasma edge. It is not yet clear for 
example whether SYCL, Kokkos or OpenMP~5 will offer the most performance 
portable and sustainable solution for \nep.
The team is therefore expected to be agile and amenable 
to change once it is clear which are the most promising long--term solutions. 
For example, a selection of SYCL for the long-term framework/code(s) would 
force refactoring of any code that is not consistent with a \nep \   library and 
code base instantiated in DPC++, and where feasible, team members 
should support this process.


Source code for all development should be accessible by the 
entire \nep \   team and all tests should be repeatable by different workers 
without the need for re-training and/or any possible confusion as to the 
procedures and metrics needed to declare a 
test successful.

\nep \   will be developed as a sequence of `core' \Papp s 
(to be distinguished from other \Papp s designed to test some novel 
technique). Core \Papp s will all need a documentation and testing framework, 
which must be agreed between all partners for the entire project. This will 
require developers to work closely with UKAEA and other team members.

A commitment is also expected by all parties to help UKAEA 
and the Met Office (as SRO for \exc) to publicise the project and build a 
fully connected community across the \exc \   programme, UKRI and Academia, 
focused upon a team of approximately twenty UK Fusion use case experts. This will 
be essential for meeting the grand challenge goal of developing a 
state-of-the-art, Exascale targeted, UK-based plasma physics simulation 
capability for the tokamak edge plasma (see Science Plan~\cite{sciplan}). 



All Core \Papp s and related infrastructure/documentation 
across the \nep \   project should meet the demands of the Code Structure and 
Coordination work package FM-WP4 in so far as the developing project standards:
\begin{itemize}
\item[$\bullet$] adopt a consistent choice of definitions (ontology) of 
objects or equivalently classes,
\item[$\bullet$] adhere to clearly defined common file formats and 
interfaces to components for data input and output.
\item[$\bullet$] provide suitably flexible data structures for common 
use by all developers,
\item[$\bullet$] are established through good scientific software 
engineering best practice,
\item[$\bullet$] demonstrate performance portability and exploit agreed 
DSL-like interfaces where possible targeting Exascale-relevant architectures,
\item[$\bullet$] can be integrated into a VVUQ framework and
\item[$\bullet$] are embedded within a coordination and benchmarking 
framework for correctness testing and performance evaluation.
\end{itemize}

In order to meet Strategic Priorities Fund terms around eligibility, 
and to steer the project towards a modular platform where developments 
across all partners can be integrated into an eventual code or platform 
available for open use by the European fusion community, a requirement 
is that all \exc \ \nep \ Grant beneficiaries make technology / source 
code developed through the programme (foreground Intellectual Property) available as open 
source under a programme--wide permissive license (currently selected as 
3-clause BSD for core/foundational infrastructure~\cite{bsd3clause}). Government 
Digital Service guidance (to which the project subscribes), discussing 
the benefits of open versus closed technology/software/data can be found 
in refs~\cite{os,os2}.
