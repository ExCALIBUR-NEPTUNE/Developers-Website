\subsection{Introduction}\label{sec:RB2_intro}
This set of requirements is based on four main sources, namely
\begin{itemize}
\item presentation by Chris Jones~(CJ) at the \nep\ internal workshop on 16 December~2019
\item points made in an email by Michael Kovari~(MK) dated 19 December~2019
\item interview with Zsolt Vizvary~(ZV) on 20 December 2019
\item use by author~(WA) of \F{SMARDDA-PFC} software in tokamak reactor design 2016-2020
\end{itemize}

A significant part of CJ's talk concerned calculations of stress in `contact' problems and in
tokamak-relevant materials, particularly those to be used in and adjacent to the
first wall. Mention was also made of nuclear heating and activation effects in the wall, and
their effect on its strength. Since \nep \ is primarily intended as a plasma modelling tool, this
material is neglected here except insofar as it has implications for \nep \ interfaces.
Similarly ZV has as high priority, a capability to do magnetic equilibrium calculations so
as to be able to calculate electromagnetic stresses in the wall.
CJ emphasised that engineering design at UKAEA makes heavy use of the ANSYS$^{TM}$
software for multiphysics calculations, and that since plasma effects including neutral beams
and fast ions are not part of any standard finite element toolkit, particular consideration should 
be given to interfacing \nep \ to ANSYS macros and ACT~(Python scripting for ANSYS).
CJ stated that ANSYS had been implementing additional physics very slowly, on a timescale of decades,
and there is anyway not a satisfactory licence model for using ANSYS on HPC,
nor any to be expected soon.

Although the \F{SMARDDA} software takes as input general surface triangulations, which may therefore
represent arbitrary surface topologies, \F{SMARDDA-PFC} calculates power deposition on the basis
of a simple, empirical physical model of transport from the outer midplane.
The errors in the model are frequently unquantifiable, given an absence of relevant experimental data,
notably when `small' limiters are proposed, where `small' implies that many lines of magnetic field make several
mid-plane passes before interacting with the limiter(s). The \nep \ software should provide a better
physical model to enable (1) calculation of the midplane profile of power `deposition' and associated
parameters such as the e-folding length~$\lambda_q$, when an exponential is fitted, and
(2) an assessment of the accuracy of the whole \F{SMARDDA} surrogate model in a wide range of
existing and novel configurations.
 

\subsection{Overall Capabilities}\label{sec:RB2_overall}
Chris Jones has been led to dream of a \\
\emph{Whole-system full-physics digital twin available for real time in-silicon simulation and experimentation}\\
but recognised that a more immediately realisable prospect was integration of plasma software into ANSYS
Workbench$^{TM}$.

\begin{enumerate}
\item Simulations must be of known, improved accuracy, so that the performance of the
built designs can be enhanced without sacrificing safety.
Thus the code and data structure should support mesh convergence studies, and be `physics aware' as described
in \Sec{RB2_physmod}.
\item The power load should be easily transferable to ANSYS for stress, heat transfer and other  multi-physics analysis.
\item The software should be easy to couple to other physics packages produced within the fusion
community such as \F{RACLETTE} and \F{ERO} for erosion calculations, \F{LOCUST} and \F{SMARDDA-PFC} for power
deposition by particles, and neutronics software such as \F{MCNP}.
\item The software should be designed to support easy production of statistics from ensemble calculations,
even when costs limit the size of the ensemble. The ensembles should be able to include different model
selections as well as different parameter choices.
\item The software should be easy to use on HPC, eg.\ through cloud-based services, where 
it should be resilient to  network bottlenecks.  Its performance
should scale well and  not depend on the OS used.
\item The software should be able to exploit GPGPUs.
%\item The software should be capable of calculating both (quasi-)static and transient behaviour
\item The software should use well-defined standard, open formats for both input and output of data.
\item The code should be easy to extend, without writing new code. Thus a user should be able to
extend (at least virtually) a data structure to include a new parameter, and add a new physical effect.
\item The following aspects of the software should be open: 
\begin{enumerate}
\item data structure requirements and documentation
\item code and documentation
\item test cases and their documentation
\item all results and their documentation
\end{enumerate}
\end{enumerate}

\clearpage
\subsection{Physics Model}\label{sec:RB2_physmod}
The software should be automatically `physics aware', ie.\ it should
\begin{enumerate}
\item be able to test the physical assumptions and orderings used, perhaps by calculations at randomly placed points, and
when this is impossible, to
produce output to enable an independent person to check against a separate code.
\item be intelligent enough to switch to a simpler assumption
when appropriate or when instructed to do so, for example by using classical
transport coefficients when they apply.
\item be able to test the level of detail used.  For example, the code  (1) 
could carry out an ensemble calculation using a simplified model of radiative loss,
and then check the results in a much shorter time against a fuller model using
a sampling technique, (2)  be able to identify
automatically that a fully 3-D field calculation with say 12 or 18 discrete TF
coils has produced a toroidally axisymmetric field in the vacuum vessel.
\item be able to simulate transient behaviour, being able to determine an initial approximate quasi-static solution, then
depending on the length of the simulation relative to physical timescales of interest, perform
either an implicit or an explicit calculation.
\item distinguish physical time from pseudo-time when relevant, eg.\ in an implicit calculation.
\item be able to handle incompatible timescales for bulk and local behaviour, eg.\ account for particle effects on overall flow.
\end{enumerate}
\clearpage
\subsection{Physics Capabilities}\label{sec:RB2_physcap}
The software should be able to 
\begin{enumerate}
\item compute the total power load on solid walls, due to plasma, fast ions and neutral beams
\item compute the production of impurity species by first wall melt and evaporation
\item compute heat conduction in first wall coatings in contact with the plasma
\item perform high frequency electrical and magnetic analysis, accounting for skin effects
\end{enumerate}
%\clearpage
\subsection{Geometry}\label{sec:RB2_geom}
\begin{enumerate}
\item The software should be able to account for the effect of changes to geometry caused by radiation swelling,
erosion and deposition due to plasma interaction, and corrosion.
\item The code should allow periodic toroidal boundary conditions.
\item The software should be able to handle 0-D, 1-D, 2-D and 3-D representations of the same plasma pulse,
transferring between the different representations.
\item A related example concerns the recognition of field axisymmetry as in \Sec{RB2_physmod}.
\item Convexities in the surface, even sharp corners capable of causing singularities in the magnetic and stress fields, should
be treatable by the software.
\end{enumerate}
