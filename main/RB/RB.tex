% Hewitt  - Use Cases
% Sommerville  - User requirements definition
% Smith  - Problem Statement
% Smith  - Requirements  - Introduction
% Smith  - Requirements  - Requirements  - Functional Requirements
% Smith  - Requirements  - General System Description  - User Characteristics

The largest input to the process, representing UKAEA Tokamak Science
Department appears under Technical Specification~\Sec{TS}, so that it may
be accompanied by a detailed response. Instead, there is presented an appreciation
of the properties of the plasma edge in \Sec{props}.

The Departmental input, although it deals with other aspects of the specification,
focusses more on the physical processes to be modelled and the approaches likely to
be required. The following \Sec{RB2} records interactions with UKAEA Engineers, although
a subsequent meeting clarified that the main demand for \nep\ at that time was that it be a
modular, component-based design, equally suitable for integration into ANSYS OptisLang$^{TM}$
as VECMAtk (with more detailed requirements likely to follow when the software became more developed).
Thereafter the  next \Sec{use_case}  contains use cases, followed by \Sec{general-remarks}
describing general requirements deducible from the use cases.

\newsection{Physical properties of the edge plasma}{sec:props}
%\section{Physical properties of the edge plasma}\label{sec:props}
The following scrape-off layer (SOL) parameters (including decay lengths) are for the
L-mode scrape off layer in MAST~\cite{Mi13Expe}.
For MAST,  with the standard notation,
$R_0=1.6$\,m, $B_T \approx 0.6$\,T, $I_\phi \approx 400$\,kA and
$a=0.6$\,m, values which imply that the poloidal field at the
plasma edge $B_p \approx 0.1$\,T.
The main result of the paper~\cite{Mi13Expe} is that the decay length of the power
deposition at midplane is $\lambda_q \approx 2$\,cm (range $1-3$\,cm)
and $P_{tot} \approx 350$\,kW.

Unless stated otherwise, the derived length, timescales and speeds are derived from
formulae and graphs in Wesson's book~\cite[Chap 10]{wesson}.
The derived quantities have been checked against SI formulae
in~\cite[Table 2.2]{miyamoto}, and also compared with those
listed in~\cite[Appendix]{Xu10Inte}. It is worth noting that although
the latter table describes the SOL of JET (and in addition its separatrix and pedestal),
JET values are typically within a factor of~$2$ of those for MAST, hence similar
numbers could be inferred for reactor designs.

\subsection{Typical discharge}

The edge values found experimentally are 
$T_e \approx 10$\,eV, $T_i \approx 20$\,eV, $n \approx 3 \times 10^{18}$\,m$^{-3}$.
These imply that the Coulomb logarithm $\Lambda \approx 12.5$, and
the flow speed~$U_d \approx 10^5$\,ms$^{-1}$ may be estimated using
$P_{tot}= 2 \pi R \lambda_q n (T_e+T_i) U_d$.
Sadly there appears to be no reliable determination of the neutral density~${\sf n}$.
(Note use of different font to distinguish neutral density from plasma density.)

\subsection{Length scales}

Debye length $\lambda_D \approx 10^{-5}$\,m. \\
Electron Larmor radius $\rho_{te} \approx 7 \times 10^{-5}$\,m. \\
$\rho_{ti} \approx 40 \rho_{te} \approx 4$\,mm. \\
Mean free path for electrons $\lambda_{emfp} \approx 1$\,m (parallel to field).

\subsection{Time scales}

%$\tau_e \approx 2 \times 10^{-8}$\,$s$.
Collision frequency (electrons with ions) $\nu_e \approx 3$\,MHz, $\tau_e \approx 3 \times 10^{-7}$\,s~\cite{NRLpf07}. \\
Plasma frequency $f_{pe} \approx 15$\,GHz, $\tau_{pe} \approx 7 \times 10^{-11}$\,s. \\
$f_{pi} \approx 0.4$\,GHz, $\tau_{pi} \approx 3 \times 10^{-9}$\,s. \\
Cyclotron frequency based on $B_p=0.1$\,T, $f_{ce} \approx 2.8$\,GHz, $\tau_{ce} \approx 4 \times 10^{-10}$\,s. \\
$f_{ci} \approx 1.4$\,MHz, $\tau_{ci}\approx  7\times 10^{-7}$\,s.

\subsection{Speeds}
Electron thermal $c_{se} \approx 1.2 \times 10^6$\,m$s^{-1}$. \\
$c_{si} \approx 4 \times 10^4$\,m$s^{-1}$. \\
Alfven speed using $B_T$ is $U_A \approx 10^7$\,m$s^{-1}$.

Collisionality parameter $\nu^{*}_c = \frac{q_e^4}{3 m_p^2\epsilon_0^2} L_0 n_0 /C_0^4$ \\
(note that $\frac{m_p^2\epsilon_0^2}{e^4}=\frac{1}{3}\;s^4 m^{-6}$.)\\
Taking $L_0 \approx 10$\,m, $n_0=n$.  Squared sound speed $C_0^2= T_i (|e|/m_e) (m_e/m_i)$, $C_0 \approx 3 \times 10^4$\,m$s^{-1}$, implies  \\
Collisionality parameter $\nu^{*}_c \approx 30$. \\
Peclet number $\approx 0.4 \nu^{*}_c \approx 10$, but turbulent coefficients$\approx1$\,m$^2$s$^{-1}$
will generally give a smaller value.

Resistive diffusion $\eta_d=15$\,$m^2 s^{-1} \propto T_e^{-3/2}$. \\
(Note that there is a notational clash with~$\eta$; fusion physics and astrophysics
differ by a factor~$\mu_0$, so that $\eta_d=\eta(\mbox{fusion})/\mu_0$.)

\subsection{Applicability of Fluid Models}\label{sec:applmhd}
A key requirement for fluid models is that collision times should be much less than
the timescale of interest, which as the preceding subsections show is true, except in the case of
$\tau_e$, the electron-ion collision time, and for the electrons more generally
for dynamics along the field-lines. The ion gyroradius is also uncomfortably large
compared to quantities of interest. Note that $\tau_e$ is the longest timescale
in the classical picture of approach to a single fluid picture of plasma, other
timescales, including the timescale for momentum to equilibrate, are shorter.

Single fluid MHD is widely used in astrophysics consistent with the eloquent advocacy
by Priest and Forbes~\cite[\S\,1.7]{priestforbes}. They point out that ideal MHD
is consistent with the drift ordering, despite confusion caused by the easy possibility
to misinterpret
Hazeltine and Meiss~\cite {hazeltinemeiss} on the subject. (The point is that
although MHD treats a faster timescale, it is valid on longer timescales, provided
relevant smaller/slower terms are retained.)
Moreover, SOL timescales involving filaments are fast, witnessed by the fact that
the ion gyro-frequency is used as normalisation for electrostatic models in~\cite{Mi12Simu},
which from \Sec{props} is a not too dissimilar timescale~$10^{-7}\,s$
to the Alfven timescale based on the poloidal field~($1$\,cm/$10^6 \approx 10^{-8}$\,s).
Later, Freidberg~\cite{freidberg}
showed that, at least in directions perpendicular to~${\bf B}$, the dynamical MHD equation
applies to a more general `guiding centre' plasma. The situation may be summarised
by saying that complexity lies mostly in the transport (diffusive) terms
as these attempt to account for low collisionality, finite Larmor radius (FLR) etc.

Perhaps fortunately, the terms predicted by kinetic theory will usually be small
(except for the electrical conductivity) compared to the turbulent transport
expected on the basis of both observation and theory of the SOL plasma. The
simplest way to account for turbulence is to assume ad-hoc isotropic, uniform `eddy'
diffusivities in addition to the usual fluid advection terms.
Lastly, in a simple extension of MHD, large~$\tau_e$  is accounted for by allowing
the electrons and ions to have different temperatures, consistent with observation.
Effects due to the presence of a large neutral population in the SOL
could well be significant, see next \Sec{neuts}.
However neutrals are mainly expected to act as a sink of momentum and energy.

\subsection{Effect of Neutrals}\label{sec:neuts}
Formulae for a weakly ionised plasma are given in the Plasma Formulary~\cite{NRLpf07}.
The collision cross-sections for electrons and ions respectively from~\cite{Ha91hydr}
are $\sigma_s^{e|0}= 10^{-19}$\,$m^2$ and
$\sigma_s^{i|0}= 4 \times 10^{-19}$\,$m^2$.
Hence, the collision frequencies for electrons and ions respectively are
\begin{equation}
\nu_{e{\sf n}}= 1.2 \times 10^{-13} {\sf n},\;\;\; \nu_{i{\sf n}}= 2 \times 10^{-14} {\sf n}
\end{equation}
where ${\sf n}$ is the neutral density.
(Note use of different font to distinguish neutral density from plasma density.)
If ${\sf n}=n$ is assumed, then
the corresponding SOL collision times are
\begin{equation}
\tau_{e{\sf n}}= 3 \times 10^{-6} s,\;\;\; \tau_{i{\sf n}}= 2 \times 10^{-5} s
\end{equation}
so that the number of collisions experienced by a typical SOL ion
before it hits a PFC is small.
Nonetheless, since $1/m_e \gg 1/m_i$, $D_e \gg D_i$ and the
diffusion coefficient for both electrons and ions is numerically large
\begin{equation}
D_A \approx (1+\frac{T_e}{T_i}) D_i \approx \frac{10^{23}}{{\sf n}}
\end{equation}

The parallel electrical diffusivities are different,
for electrons and ions respectively these are
\begin{equation}
\eta_{e {\sf n}\parallel} =4 \frac{{\sf n}}{n},\;\;\;
\eta_{i {\sf n}\parallel} =0.5 \frac{{\sf n}}{n}
\end{equation}
The implication from the formulae in~\cite{Le06emer} is that 
the value for $\eta_{e \parallel}$ combines additively with the usual
Spitzer value in a more
highly ionised plasma. Assuming ${\sf n} \approx n$, however the correction
is seen to be an increase of $4$ in $15$\,$m^2s^{-1}$, i.e. only about~$25$\,\%.

Arber~\cite{Le06emer,Ar07Emer} further points out that
according to the Formulary~\cite{NRLpf07},
in a weakly ionised plasma
the conductivity is greatly reduced (and the magnetic diffusivity
correspondingly enhanced) in directions normal to a strong magnetic field.
Typically for Braginskii theory, the
factor is $x_e^2$ for the perpendicular direction and $x_e$ for
the other direction, where
\begin{equation}
x_e=\frac{2\pi f_{ce}}{\nu_{e{\sf n}}} \approx \frac{8 \times 10^{23}}{{\sf n}}
\end{equation}
For ${\sf n}=n=3 \times 10^{18}$, these are huge increases. However it is worth
noting that if the electromagnetic potential representation is invoked, so that
\begin{equation}\label{eq:E}
{\bf E} = -\nabla \Phi + \frac{\partial {\bf A}}{\partial t}
\end{equation}
then, in the direction parallel to~${\bf B}$, neglecting the gradient of
electric potential~$\Phi$
\begin{equation}\label{eq:A}
\frac{\partial  A_{\parallel}}{\partial t}=\eta_{e {\sf n}\parallel} J_{\parallel}
\end{equation}
Thus the enhanced diffusivities need not signify if this equation is used for
magnetic field evolution, although applying a gauge condition on the potentials
may become difficult in complicated 3-D topologies.



\newsection{Engineering Requirements Baseline}{sec:RB2}
\subsection{Introduction}\label{sec:RB2_intro}
This set of requirements is based on four main sources, namely
\begin{itemize}
\item presentation by Chris Jones~(CJ) at the \nep\ internal workshop on 16 December~2019
\item points made in an email by Michael Kovari~(MK) dated 19 December~2019
\item interview with Zsolt Vizvary~(ZV) on 20 December 2019
\item use by author~(WA) of \F{SMARDDA-PFC} software in tokamak reactor design 2016-2020
\end{itemize}

A significant part of CJ's talk concerned calculations of stress in `contact' problems and in
tokamak-relevant materials, particularly those to be used in and adjacent to the
first wall. Mention was also made of nuclear heating and activation effects in the wall, and
their effect on its strength. Since \nep \ is primarily intended as a plasma modelling tool, this
material is neglected here except insofar as it has implications for \nep \ interfaces.
Similarly ZV has as high priority, a capability to do magnetic equilibrium calculations so
as to be able to calculate electromagnetic stresses in the wall.
CJ emphasised that engineering design at UKAEA makes heavy use of the ANSYS$^{TM}$
software for multiphysics calculations, and that since plasma effects including neutral beams
and fast ions are not part of any standard finite element toolkit, particular consideration should 
be given to interfacing \nep \ to ANSYS macros and ACT~(Python scripting for ANSYS).
CJ stated that ANSYS had been implementing additional physics very slowly, on a timescale of decades,
and there is anyway not a satisfactory licence model for using ANSYS on HPC,
nor any to be expected soon.

Although the \F{SMARDDA} software takes as input general surface triangulations, which may therefore
represent arbitrary surface topologies, \F{SMARDDA-PFC} calculates power deposition on the basis
of a simple, empirical physical model of transport from the outer midplane.
The errors in the model are frequently unquantifiable, given an absence of relevant experimental data,
notably when `small' limiters are proposed, where `small' implies that many lines of magnetic field make several
mid-plane passes before interacting with the limiter(s). The \nep \ software should provide a better
physical model to enable (1) calculation of the midplane profile of power `deposition' and associated
parameters such as the e-folding length~$\lambda_q$, when an exponential is fitted, and
(2) an assessment of the accuracy of the whole \F{SMARDDA} surrogate model in a wide range of
existing and novel configurations.
 

\subsection{Overall Capabilities}\label{sec:RB2_overall}
Chris Jones has been led to dream of a \\
\emph{Whole-system full-physics digital twin available for real time in-silicon simulation and experimentation}\\
but recognised that a more immediately realisable prospect was integration of plasma software into ANSYS
Workbench$^{TM}$.

\begin{enumerate}
\item Simulations must be of known, improved accuracy, so that the performance of the
built designs can be enhanced without sacrificing safety.
Thus the code and data structure should support mesh convergence studies, and be `physics aware' as described
in \Sec{RB2_physmod}.
\item The power load should be easily transferable to ANSYS for stress, heat transfer and other  multi-physics analysis.
\item The software should be easy to couple to other physics packages produced within the fusion
community such as \F{RACLETTE} and \F{ERO} for erosion calculations, \F{LOCUST} and \F{SMARDDA-PFC} for power
deposition by particles, and neutronics software such as \F{MCNP}.
\item The software should be designed to support easy production of statistics from ensemble calculations,
even when costs limit the size of the ensemble. The ensembles should be able to include different model
selections as well as different parameter choices.
\item The software should be easy to use on HPC, eg.\ through cloud-based services, where 
it should be resilient to  network bottlenecks.  Its performance
should scale well and  not depend on the OS used.
\item The software should be able to exploit GPGPUs.
%\item The software should be capable of calculating both (quasi-)static and transient behaviour
\item The software should use well-defined standard, open formats for both input and output of data.
\item The code should be easy to extend, without writing new code. Thus a user should be able to
extend (at least virtually) a data structure to include a new parameter, and add a new physical effect.
\item The following aspects of the software should be open: 
\begin{enumerate}
\item data structure requirements and documentation
\item code and documentation
\item test cases and their documentation
\item all results and their documentation
\end{enumerate}
\end{enumerate}

\clearpage
\subsection{Physics Model}\label{sec:RB2_physmod}
The software should be automatically `physics aware', ie.\ it should
\begin{enumerate}
\item be able to test the physical assumptions and orderings used, perhaps by calculations at randomly placed points, and
when this is impossible, to
produce output to enable an independent person to check against a separate code.
\item be intelligent enough to switch to a simpler assumption
when appropriate or when instructed to do so, for example by using classical
transport coefficients when they apply.
\item be able to test the level of detail used.  For example, the code  (1) 
could carry out an ensemble calculation using a simplified model of radiative loss,
and then check the results in a much shorter time against a fuller model using
a sampling technique, (2)  be able to identify
automatically that a fully 3-D field calculation with say 12 or 18 discrete TF
coils has produced a toroidally axisymmetric field in the vacuum vessel.
\item be able to simulate transient behaviour, being able to determine an initial approximate quasi-static solution, then
depending on the length of the simulation relative to physical timescales of interest, perform
either an implicit or an explicit calculation.
\item distinguish physical time from pseudo-time when relevant, eg.\ in an implicit calculation.
\item be able to handle incompatible timescales for bulk and local behaviour, eg.\ account for particle effects on overall flow.
\end{enumerate}
\clearpage
\subsection{Physics Capabilities}\label{sec:RB2_physcap}
The software should be able to 
\begin{enumerate}
\item compute the total power load on solid walls, due to plasma, fast ions and neutral beams
\item compute the production of impurity species by first wall melt and evaporation
\item compute heat conduction in first wall coatings in contact with the plasma
\item perform high frequency electrical and magnetic analysis, accounting for skin effects
\end{enumerate}
%\clearpage
\subsection{Geometry}\label{sec:RB2_geom}
\begin{enumerate}
\item The software should be able to account for the effect of changes to geometry caused by radiation swelling,
erosion and deposition due to plasma interaction, and corrosion.
\item The code should allow periodic toroidal boundary conditions.
\item The software should be able to handle 0-D, 1-D, 2-D and 3-D representations of the same plasma pulse,
transferring between the different representations.
\item A related example concerns the recognition of field axisymmetry as in \Sec{RB2_physmod}.
\item Convexities in the surface, even sharp corners capable of causing singularities in the magnetic and stress fields, should
be treatable by the software.
\end{enumerate}


\newsection{Use Cases}{sec:use_case}
\subsection{Use Cases: Tokamak edge physicist}\label{sec:edge-boundary-tokamak-physicist}

They are early career, and to progress they need to build a professional
reputation by publishing papers, supporting UKAEA's research programmes
and supervising students. They are a competent developer and experienced
HPC user, though they do not gain any credit directly from developing
software.

In their research work, they study different models for the tokamak
edge, and so require code flexibility and a user-friendly DSL to allow
them to rapidly prototype different equation sets. This work would
require quick iterations -- perhaps 5 minute simulations performed on a
desktop. They will also develop their own algorithms and add
infrastructure to the code. While they will do this with an
understanding of performance implications, they would expect to perform
these developments at a higher level that raw performance loops (but at
a lower level than the physics model).

They would expect to contribute their changes back to a community
repository, and also to benefit from changes that other code users have
made. They would be involved in the community -- perhaps raising issues,
making and reviewing git pull requests, answering queries, and having input into
future code releases -- but would not be involved ``project management''
tasks, like maintaining the repository.

They will also value a user-friendly interface and active user community
when it comes to working with their students. In this context it is
valuable to have software that will run at a high level and produce
sensible results without needing to specify the details of the
implementation. This allows the student to learn about physical systems
without simultaneously having to learn the details of numerical
implementations. The active community allows their student to get
support and ask (perhaps trivial) questions without being dependent on
their supervisor.

Finally, in support of experiments, they will need to perform
high-fidelity simulations of tokamaks. These will be highly
computationally expensive, either because they are high-resolution
simulations of specific shots, or because they are parameter scans or UQ
campaigns. The simulations will be long-running, perhaps in the range of
a week to a few months, on whichever HPC system that they have access
to. The software must therefore be performance portable in order to
facilitate high performance on a range of systems. The software also
needs to be robust to numerical instabilities, hardware node failures,
etc, as one may not have the resource allocation to repeat failed runs.

\subsection{Use Cases: Engineers}\label{sec:use-cases-engineers}

\emph{As a thermomechanical engineer I:}

\begin{itemize}
\item work with a large range of open source and proprietary codes which
  requires bindings to other tools, eg\.:

  \begin{itemize}
  \item FMU
  \item Python (Jupyter and regular)
  \item OptiSLang
  \item Twinbuilder
  \end{itemize}
\item work with CAD software to generate geometries which I want to
  propagate through my workflow.
\item am interested in heat fluxes in all forms: from time and space
  averages to high resolution 3D time and data.
\item need to be able iterate on designs quickly and in an automated way.
\item am neither an HPC expert nor a plasma / tokamak physicist.
\end{itemize}

\emph{and I want to:}

\begin{itemize}
\item know, given a sensible physics model provided by other experts, what
  the transient peak and average heat loads are on plasma facing
  components.
\item not have to understand software dependencies and be able to install and
  run easily eg.\ ``in the cloud''.
\item be able to configure the software to undertake parameter scans.
\item have a handle on the sensitivity of the solution to the inputs and
  sources of error / uncertainty.
\item be able to re-use the spatio-temporal heat fluxes as a model in more
  thermo-mechanical calculations. This means:

  \begin{itemize}
  \item reading in the solution after the calculation.
  \item using a fit to the data in the form of eg.\ a reduced order model of
    the heat fluxes, surrogates etc.
  \item being able to export CAD geometries and import solutions back into
    engineering tools eg.\ ANSYS.
  \end{itemize}
\item have reproducible workflows to save and share with colleagues
  (eg.\ databases of inputs / outputs / config).
\item export results flexibly to inter-operate with multiple surrogate
  frameworks.
\item be in the loop with development process so it is possible to keep other
  workflows up to date.
\end{itemize}

\emph{This would mean I can:}

\begin{itemize}
\item design components with colleagues within eg.\ STEP.
\item make use of the \nep\ software in combination with proprietary tools that engineers
  know inside out.
\item be insulated, to a sensible extent, from the complexities of the
  numerics, plasma physics and HPC.
\end{itemize}

\subsection{Use Cases: Particle Specialists}\label{sec:use-cases-particle-specialists}

\emph{As a particle specialist I:}

\begin{itemize}
\item am very familiar with particle based methods.
\item may not be familiar with FEM implementation details but have a working
  understanding of the approach.
\item may not be familiar with low level languages.
\item may not be familiar with HPC hardware and architectures.
\item understand how to describe complex physical processes such as
  radiation, recombination, ionisation, charge exchange using both
  particle and FEM data.
\item may not have applied UQ techniques before but may have an
  understanding of distributions/ensembles from statistical mechanics.
\end{itemize}

\emph{I want to:}

\begin{itemize}
\item describe particle based operations both collectively and per particle,
  eg.\ :
\begin{itemize}
\item creation and deletion of particles potentially from complex
  distributions.
\item computation with particle data - per particle and collectively.
\item identification of groups of particles.
\item representation of arbitrary per particle data.
\end{itemize}
\item visualise particle and FEM data - snapshots and trajectories.
\item create new finite element functions on appropriate function spaces.
\item add source/sink terms to governing equations (solved with FEM).
\item define particle source and sink regions using the simulation domain
  geometry.
\item define regions of interest, eg.\ surfaces, as part of diagnostics.
\item identify particles near surfaces/points/volumes of interest.
\item create and use global data structures for computation,
  eg.\ diagnostics.
\item represent particle data as FE functions:

  \begin{enumerate}
  \def\labelenumi{\arabic{enumi}.}
  \item through pointwise projection.
  \item line integration over particle trajectory.
  \end{enumerate}
\item use non-trivial functions in my loops, eg.\ erfc, gamma.
\item define functions using expansion coefficients, eg.\ ionisation rate
  function approximated by an exponential expansion.
\item evaluate these functions using both particle and FEM data.
\item describe pairwise operations that implement physical processes.
\item describe and sample from non-trivial statistical distributions.
\item perform simulations in a reproducible manner.
\end{itemize}

\emph{So that I can:}

\begin{itemize}
\item use particles as a kinetic description for plasma and neutrals.
\item represent highly-collisional regimes by fluid approximations.
\item describe plasma-neutral and plasma-plasma interactions.
\item use abstractions/DSLs to write once, run anywhere as much as possible.
\item experiment with models quickly and efficiently.
\item perform ensemble computations and averages.
\item perform UQ and verification.
\end{itemize}

%This is an attempt at a profile of a certain class of \nep \   user stating 
%what might be reasonable expectations of the code and its interface.
\subsection{Use Cases: Finite Element Background}\label{sec:use-cases-finite-element-background}

I am a user with perhaps some grasp of plasma physics but with a more extensive 
knowledge of finite-element software (I might be an experienced user / 
developer of Nektar++).  My background may be either physics or engineering; I 
may be a new recruit to the \nep \ team and needing to learn the code with a 
view to taking a future role as a \nep \ developer.

I need the interface / DSL to provide access to typical FEM parameters eg.\ 
choice of intra-element basis functions and their polynomial order, continuous 
/ discontinuous Galerkin, choice of numerical flux, stabilization options; also 
whether diffusion and advection terms are explicit or implicit.  In line with 
eg.\ Nektar++ I expect the choice of time-stepper to be largely ``orthogonal'' 
to most details mentioned above (the exception is explicit / implicit choice).  
I would like the option to specify the timestep in terms of the CFL number.
In addition I require control 
over relevant meshing parameters eg.\ element spatial density and approximation 
order of any curvilinear elements.
I would like the DSL to be able to generate a range of regular meshes 
internally (at least for trivial cases eg.\ boxes meshed with quads).

I should like some simple, physically-motivated canonical examples that might
assist with learning plasma physics.

I expect the performance of the code to be at least commensurate with other FEM 
packages eg.\ Nektar++ and to remain so going forward (and obviously must be 
scalable to the latest hardware, which means foreseeably an efficient GPU 
implementation, supporting ideally NVidia, AMD, and Intel Xe / Ponte Vecchio).

I am unused to velocity-space effects.
I would like the particles aspects of the code to be expressible, insofar as is 
possible, in FEM language: the conversion from discrete to continuum should 
ideally not be visible to me eg.\ converting particles to FEM forcing terms.  
Further to this, it would be good if a set of default particle parameters could 
be produced based on FEM parameters, as required (perhaps a reasonable value 
for the number of particles can be derived automatically based on FEM 
resolution).

If there is an issue from PIC compatibility (eg.\ constrained choice of basis 
functions), the DSL should make this clear in an explicit error message, plus 
hopefully advice how to remedy.

%The interface to higher-dimensional kinetic aspects needs careful thought.



\newsection{General Remarks}{sec:general-remarks}
Particular, important general aspects of the use cases and \Sec{TS} requirements may be
stated as follows:

Calculations may need be (re)started, perhaps from databases of calculations as envisaged by the 
IMAS development.  Someone from the experimental side might want to specify input parameters
by duplicating those of a particular say JET 
shot at a given time, using an database of experimental results.
Physicists of either stamp (theoretical or experimental) will likely want compatibility with
analysis software such as OMFIT~\cite{omfitwebsite} and tools to speed publication in
the scientific literature of results obtained.
An engineer may simply want to ``change surface A to another design and repeat calculation".

Generally, minimising the number of new languages and systems people have to learn, especially
in view of the need to attract people to the project and community, seems a good idea, so that
a Domain Specific Language (DSL) should for
example be based on one or more language(s) that are already well-known to many technical people,
such as Julia, Python or \LaTeX.
Equally \nep \ software cannot be allowed to ossify, so the suggestion 
is everywhere to have a preferred option and an allowed option where this makes sense.
