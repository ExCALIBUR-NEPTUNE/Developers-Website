\begin{table}[h]
\begin{center}
\caption{Key Strategies for Object Identification. After Table~5.1 from~\cite{douglass},
slightly amended.
All the strategies except the last, are concerned with identifying the objects listed.
\label{tab:objdisc}}
\begin{tabular}{|p{3.5cm}|p{12.0cm}|}
\hline
Strategy &  Description \\
\hline
\hline
Nouns  &  Used to gain a first-cut object list, the 
analyst underlines each noun or noun phrase in the problem statement and 
evaluates it as a potential object, class, or attribute. \\
\hline
Causal agents &  Identify the sources of actions, events, and messages; includes the 
coordinators of actions. \\
\hline
Services (passive contributors) &  Identify the 
targets of actions, events, and messages as well as entities that passively 
provide services when requested. \\
\hline
Messages and information flow &  Messages 
must have an object that sends them and an object that receives them as well 
as, possibly other objects that process the information contained in the messages.
There are many ways to identify the objects within a collaboration.  \\
\hline
Real-world items &  Real-world items are entities that 
exist in the real world, but are not necessarily electronic devices.  Examples 
include objects such as gases, forces,
blanket modules, etc. \\
\hline
Physical devices &  Physical devices include the sensors and actuators provided by the 
system as well as the electronic devices they monitor or control.  The resulting
objects are almost always the interfaces to the devices.
Note: this is a special kind of ``Identify real-world items". \\
\hline
Key concepts &  Key concepts may be modeled as objects.  Physical theories exist
only conceptually, but are critical scientific objects.  Frequency bins
for an on-line autocorrelator may also be objects. Contrast with
the ``identify real-world items" strategy. \\
\hline
Transactions &  Transactions are finite instances of interactions 
between objects that persist for some significant period of time. An example
is queued data. \\
\hline
Persistent information &  Information that must 
persist for significant periods of time may be objects or attributes.  This persistence
may extend beyond the power cycling of the device. \\
\hline
Visual elements &  User-interface elements that display data are objects within the 
user-interface domain such as windows, buttons, scroll bars, menus, histograms, 
waveforms, icons, bitmaps, and fonts. \\
\hline
Control elements &  Control elements are objects that provide the interface
for the user (or some external device) to control system behavior. \\
\hline
Apply scenarios &  Walk through scenarios using the identified objects. Missing objects
will become apparent when required actions 
cannot be achieved with existing objects and relations. \\
\hline
\hline
\end{tabular}
\end{center}
\end{table}
